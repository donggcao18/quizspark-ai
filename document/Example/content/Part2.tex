\chapter{CHƯƠNG TRÌNH KIỂM TRA CÚ PHÁP LỆNH MIPS}
\begin{center}
    \textbf{Sinh viên thực hiện: Trần Quang Huy - 20226109}
\end{center}
\section{Đề bài}
Trình biên dịch của bộ xử lý MIPS sẽ tiến hành kiểm tra cú pháp các lệnh hợp ngữ trong mã nguồn, xem có phù hợp về cú pháp hay không, rồi mới tiến hành dịch các lệnh ra mã máy. Hãy viết một chương trình kiểm tra cú pháp của 1 lệnh hợp ngữ MIPS bất kì (không làm với giả lệnh) như sau:
\begin{itemize}
    \item Nhập vào từ bàn phím một dòng lệnh hợp ngữ. Ví dụ beq s1,31,t4.
    \item Kiểm tra xem mã opcode có đúng hay không? Trong ví dụ trên, opcode là beq là hợp lệ thì hiện thị thông báo “\textit{\bf opcode: beq, hợp lệ}”
    \item Kiểm tra xem tên các toán hạng phía sau có hợp lệ hay không?
\end{itemize}
Trong ví dụ trên, toán hạng $s1$ là hợp lệ, 31 là không hợp lệ, t4 thì khỏi phải kiểm tra nữa vì toán hạng trước đã bị sai rồi.
Gợi ý: nên xây dựng một cấu trúc chứa khuôn dạng của từng lệnh với tên lệnh, kiểu của toán hạng 1, toán hạng 2, toán hạng 3.

\section{Định hướng cách làm}
\subsection{Quy trình tổng thể}
\begin{enumerate}
    \item \textbf{Khởi tạo}
    \begin{itemize}
        \item \textbf{Lưu trữ opcode và thanh ghi hợp lệ}
        Chúng ta cần lưu các opcode và thanh ghi hợp lệ vào một vị trí cụ thể. Điều này cho phép kiểm tra tính hợp lệ của các lệnh nhập vào. Sử dụng mảng ký tự để lưu trữ các opcode mẫu và các thanh ghi mẫu, các giá trị này được phân cách bằng dấu ‘/’ và kết thúc bằng dấu cách.
        \item \textbf{Chia opcode thành các nhóm:}
        Việc chia opcode thành các nhóm không chỉ dựa trên việc chúng thuộc khuôn dạng R, I hay J mà còn dựa trên khuôn dạng lệnh khi code.\\
        Ví dụ: \texttt{addi} và \texttt{beq} thuộc loại I nhưng khuôn dạng code là khác hẳn nhau:
\begin{itemize}
    \item \texttt{addi \$s1, \$s2, 2}
    \item \texttt{beq \$s1, \$s2, LoopA}
\end{itemize}
    \end{itemize}
    \item \textbf{Logic chính}
    \begin{itemize}
        \item Tách mã opcode từ dòng lệnh đầu vào.
        \item Kiểm tra tính hợp lệ của opcode so với các tập lệnh đã được định trước.
        \item Dựa trên loại opcode (R, I, J, L), tiếp tục tách và xác thực các thanh ghi và số trong lệnh
    \end{itemize}
    \item \textbf{Gọi hàm và xác thực}
    \begin{itemize}
        \item \textbf{Gọi hàm chính}
        \begin{itemize}
    \item \textbf{\tt Split\_opcode}: Tách phần opcode và lưu vào hàng đợi.
    \item \textbf{\tt Split\_Register\_and\_Number}: Tách phần thanh ghi hoặc số và lưu vào hàng đợi.
    \item \textbf{\tt Split\_Sign\_ExtImm}: Tách ký tự dùng để tách khuôn dạng \texttt{0(\$s2)} và lưu vào hàng đợi.
    \item \textbf{\tt Check\_opcode}: Kiểm tra opcode có hợp lệ hay không và xác định khuôn dạng của nó, sau đó gán giá trị tương ứng cho \texttt{\$s4}:
    \begin{itemize}
        \item R (\texttt{\$s4 = 1})
        \item R\_1 (\texttt{\$s4 = 2})
        \item R\_2 (\texttt{\$s4 = 6})
        \item I (\texttt{\$s4 = 3})
        \item I\_1 (\texttt{\$s4 = 7})
        \item J (\texttt{\$s4 = 4})
        \item J\_1 (\texttt{\$s4 = 8})
        \item L (\texttt{\$s4 = 9})
        \item L\_1 (\texttt{\$s4 = 10})
        \item Đặc biệt (syscall, nop $\rightarrow$ \texttt{\$s4 = 5})
        \item Opcode sai (\texttt{\$s4 = 0})
    \end{itemize}
\end{itemize}
\item \textbf{Kiểm tra và nhảy tới hàm phù hợp}
Ứng với từng giá trị của \texttt{\$s4}, chúng ta sẽ nhảy tới hàm kiểm tra tương ứng với khuôn dạng đó \texttt{(R, R1, I, ...)}.

    \end{itemize}
    \item \textbf{Lặp lại và thoát}
    \begin{itemize}
        \item \textbf{Hiển thị kết quả}
        Sau khi kiểm tra, chúng ta sẽ in ra kết quả rằng câu lệnh nhập vào có cấu trúc đúng hay sai.
        \item \textbf{Lặp lại kiểm tra}
        Sau khi hiển thị kết quả, hỏi người dùng có muốn kiểm tra tiếp không. Nếu có, lặp lại quy trình từ bước tách và kiểm tra câu lệnh mới. Nếu không, thoát khỏi chương trình.
    \end{itemize}
\end{enumerate}
\subsection{Lưu đồ thuật toán}

\begin{figure}[H]
	\begin{framed}
		\centering
		\includegraphics[width=11.0cm]{images/luudo.png}
		\caption{Lưu đồ thuật toán}
		\label{s2attn}
	\end{framed}
\end{figure}
\newpage
\section{Thuật toán và mã nguồn từng bước}
\subsection{Khai báo dữ liệu}

\begin{lstlisting}[]
.data
Message1: .asciiz "Nhap dong lenh can check: "
Message2: .asciiz "\nOpcode: "
Message3: .asciiz ", hop le!"
Message4: .asciiz " khong hop le!"
Message5: .asciiz " \nCau lenh dung!\n-------------------\n"
Message6: .asciiz " \nCau lenh sai!\n-------------------\n"
Message7: .asciiz " \n"
Message8: .asciiz "Thanh ghi "
Message9: .asciiz "So "
Message10: .asciiz "Nhan "
Message11: .asciiz "Ban muon kiem tra tiep khong?"
string: .space 100
#Luu cac opcode can check vao mang
Opcode_R_Check: .asciiz "/add/sub/addu/subu/and/or/slt/sltu/nor/srav/srlv/movn/movz/mul/ "
Opcode_R_Check_1: .asciiz "/beq/bne/ "
Opcode_R_Check_2: .asciiz "/div/divu/mfc0/mult/multu/clo/clz/move/negu/not/madd/maddu/msub/msubu/ "
Opcode_I_Check: .asciiz "/addi/addiu/andi/ori/slti/sltiu/sll/srl/sra/ "
Opcode_I_Check_1: .asciiz "/li/lui/ "
Opcode_J_Check: .asciiz "/j/jal/ "
Opcode_J_Check_1: .asciiz "/jr/mfhi/mthi/mflo/mtlo/ "
Opcode_L_Check: .asciiz "/lb/lbu/lh/lhu/ll/lw/sb/sc/sh/sw/lwc1/ldc1/swc1/sdc1/ "
Opcode_L_Check_1: .asciiz "/la/ "
Special_command: .asciiz "/syscall/nop/ "
Register_Check: .asciiz "/$zero/$at/$v0/$v1/$a0/$a1/$a2/$a3/$t0/$t1/$t2/$t3/$t4/$t5/$t6/$t7/$s0/$s1/$s2/$s3/$s4/$s5/$s6/$s7/$t8/$t9/$k0/$k1/$gp/$sp/$sp/$fp/$ra/$0/$1/$2/$3/$4/$5/$6/$7/$8/$9/$10/$11/$12/$13/$14/$15/$16/$17/$18/$19/$20/$21/$22/$23/$24/$25/$26/$27/$28/$29/$30/$31/ "
chain\_check: .word 	# Chua xau ki tu dang xet
.text
start:
	la	$s2, chain_check #Dia chi chua chain_check
	li	$s6, 32		#s6=space
	li	$s7, 47		#s7 = '/'
#Nhap dong lenh can check
	li 	$v0, 54
	la 	$a0, Message1
	la 	$a1, string
	la 	$a2, 100
	syscall
	la 	$s1, string

\end{lstlisting}
\begin{enumerate}
    \item \textbf{Khai báo thông báo dạng chuỗi}

\texttt{Message1, Message2, ..., Message11}: Đây là các thông báo dạng chuỗi sử dụng .asciiz để khai báo. Các thông báo này sẽ được sử dụng trong chương trình để hiển thị thông tin cho người dùng.
\item \textbf{Khai báo danh sách opcode để kiểm tra}


\texttt{Opcode\_R\_Check}, \texttt{Opcode\_R\_Check\_1}, \ldots, \texttt{Opcode\_L\_Check\_1}, \texttt{Special\_command}: Đây là các chuỗi chứa các opcode hợp lệ để chương trình sử dụng để kiểm tra tính hợp lệ của các lệnh người dùng nhập vào.
\end{enumerate}
\subsection{Chương trình chính}
\begin{lstlisting}[]
#main
	jal    Split_opcode
	jal	   Check_opcode
	beq	   $s4, $zero, False_opcode	#Opcode false
	addi   $t0, $zero, 5		#Syscall, nop->Right code
	beq	   $s4, $t0, Right_code
	addi   $t5, $zero, 1
	beq	   $s4, $t5, R_Check_Register_and_Number
	addi   $t5, $zero, 2
	beq	   $s4, $t5, R_1_Check_Register_and_Number
	addi   $t5, $zero, 3
	beq    $s4, $t5, I_Check_Register_and_Number
	addi   $t5, $zero, 4
	beq	   $s4, $t5, J_Check_Register_and_Number
	addi   $t5, $zero, 6
	beq	   $s4, $t5, R_2_Check_Register_and_Number
	addi   $t5, $zero, 7
	beq    $s4, $t5, I_1_Check_Register_and_Number
	addi   $t5, $zero, 8
	beq	   $s4, $t5, J_1_Check_Register_and_Number
	addi   $t5, $zero, 9
	beq	   $s4, $t5, L_Check_Register_and_Number
	addi   $t5, $zero, 10
	beq	   $s4, $t5, L_1_Check_Register_and_Number
	j  End_main
\end{lstlisting}
\textbf{Giải thích chương trình chính}
\begin{itemize}
\item Tách opcode từ dòng lệnh nhập vào, kiểm tra xem opcode nhập vào là dạng nào (R,I,J,L, đặc biệt).
    \item Kiểm tra xem \texttt{\$s4} đúng hay sai, nếu sai chạy tới hàm \texttt{False\_opcode} và in ra màn hình opcode sai. Nếu \texttt{\$s4=0}, đó là \texttt{syscall}, nhảy tới \texttt{Right\_opcode}.
    \item Sau đó là câu lệnh so sánh \texttt{\$s4} với lần lượt với tham số 1,2,3,4. Với mỗi opcode, nhảy tới hàm check Thanh ghi và Số hạng tương ứng.
    \item Trong trường hợp không khớp với bất kì loại opcode, nhảy tới \texttt{End main}.
\end{itemize}

\subsection{Tách Opcode và Kiểm tra, phân loại Opcode}
\subsubsection{a. Chức năng một số thanh ghi được sử dụng}
\begin{table}[H]
\centering
\begin{tabular}{|c|l|}
\hline
\textbf{Thanh ghi} & \textbf{Ý nghĩa} \\ \hline
\texttt{\$s0} & Vị trí phần tử cuối của hàng đợi \\ \hline
\texttt{\$s1} & Địa chỉ của lệnh đầu vào \\ \hline
\texttt{\$s2} & Địa chỉ của hàng đợi (phần tử đầu tiên) \\ \hline
\texttt{\$s4} & Lưu giá trị tương ứng với opcode của lệnh \\ \hline
\texttt{\$s5} & Vị trí đang load từ \texttt{\$s1} (ví dụ: “beq \texttt{\$s1}, \texttt{\$s2}, ABC” thì \texttt{\$s5} = 4) \\ \hline
\texttt{\$t9} & Kí tự cuối cùng được load \\ \hline
\texttt{\$s6} & Giá trị 32, tương ứng với kí tự khoảng trắng (space) \\ \hline
\texttt{\$s7} & Giá trị 47, tương ứng với kí tự '/' \\ \hline
\texttt{\$s3} & Địa chỉ của chuỗi opcode mẫu (đã lưu từ trước) \\ \hline
\texttt{\$a2} & Địa chỉ của kí tự opcode đang được load từ hàng đợi cần kiểm tra \\ \hline
\texttt{\$a3} & Địa chỉ của kí tự đang được load của opcode mẫu \\ \hline
\texttt{\$s4} & Chứa giá trị của khuôn dạng lệnh (mặc định ban đầu là 0) \\ \hline
\texttt{\$t1} & Biến đếm i (mặc định ban đầu là 0) \\ \hline
\texttt{\$t2} & Kí tự được load từ \texttt{\$a2} \\ \hline
\texttt{\$t3} & Kí tự được load từ \texttt{\$a3} \\ \hline
\texttt{\$t0} & Số kí tự của opcode mẫu \\ \hline
\end{tabular}
\caption{Các thanh ghi và ý nghĩa của chúng}
\label{table:registers}
\end{table}


\subsubsection{b. Hàm tách opcode}
\begin{lstlisting}
#----------------------------------------------------	
#Tach ma opcode
Split_opcode:
#Khoi tao:
	li	$s5, 0		#Vi tri load ban dau cua lenh nap vao
	li	$s0, 0		#Vi tri phan tu cuoi cua mang chain_check
	li	$t1, 0		#i=0
#Tach ky tu dau tien cua opcode, loai bo ki tu space thua
Loop1:
	add	$a2, $s1, $t1	#a2 = Dia chi cua ky tu dang load
	add	$a3, $s2, $s0	#a3 = Dia chi dang nap vao hang doi
	lb	$t0, 0($a2)
	beq	$t0, $zero, EndLoop	#Gap null => ket thuc vong lap 1
	beq	$t0, $s6, Loop1_them	#s6=space -> Xu li loai bo space
	sb	$t0, 0($a3) 		#Nap ky tu vao hang doi
	addi	$s0, $s0, 1		#Dich chuyen vi tri cuoi cua hang doi sang phai
	addi	$t1, $t1, 1		#Tang vi tri cuoi hang doi
	addi	$s5, $s5, 1		#Tang vi tri load ban dau lenh nhap vao.
#tach cac ki tu tiep theo cua opcode luu vao queue, gapnewline/space-> end
Loop2:
	add	$a2, $s1, $t1	#a2 = Dia chi cua ky tu dang load
	add	$a3, $s2, $s0	#a3 = Dia chi dang nap vao hang doi
	lb	$t0, 0($a2)
	beq	$t0, $zero, EndLoop	#Gap null => ket thuc vong lap 1
	beq	$t0, $s6, EndLoop	#Gap space => ket thuc vong lap 1
	li	$t5, 10			#t5=newline
	beq	$t0, $t5, EndLoop	#Gap newline => ket thuc vong lap 1
	sb	$t0, 0($a3) 		#Nap ky tu vao hang doi
	addi	$s0, $s0, 1		#Dich chuyen vi tri cuoi cua hang doi sang phai
	addi	$t1, $t1, 1
	addi	$s5, $s5, 1
	j	Loop2
EndLoop:
	#Chen ky tu NULL cho hang doi
	sb	$zero, 0($a3)
	add	$s5, $s0, $zero		#Luu vi tri ki tu dang doc vao s5
	addi	$s0, $s0, -1
	jr	$ra
\end{lstlisting}
\textbf{Giải thích}
\begin{itemize}
    \item \textbf{Vòng lặp \texttt{Loop1}}: Hàm giúp load kí tự đầu tiên vào các thanh ghi đồng thời để loại bỏ các kí tự space thừa ở trước opcode. Chương trình sẽ được quay lại lặp tiếp cho đến khi gặp kí tự \texttt{NULL} hoặc tất cả các ký tự đầu tiên của opcode đã được tách và lưu vào hàng đợi.
    \item \textbf{Vòng lặp \texttt{Loop2}}: Vòng lặp thứ hai nhằm load các kí tự tiếp theo của opcode và lưu vào hàng đợi. Tương tự như vòng lặp đầu tiên, nhưng vòng lặp này kiểm tra thêm các ký tự dấu cách (space) và xuống dòng (newline) để kết thúc vòng lặp nếu gặp. \texttt{Loop2} kết thúc khi gặp space hoặc newline hoặc null.
    \item \textbf{Kết thúc \texttt{(EndLoop)}}: Thêm ký tự NULL vào cuối hàng đợi để đánh dấu kết thúc chuỗi opcode. Sau đó thực hiện các thao tác lưu vị trí kí tự và trở lại hàm main.
\end{itemize}

\subsubsection{c. Hàm check opcode}
\begin{lstlisting}
#----------------------------------------------------	
#Check Opcode
Check_opcode:
	li	$s4, 0	#s4 bieu thi cho khuon dang lenh: Saiopcode: 0, R: 1, R_1: 2, I: 3, J: 4, Dac biet: 5
	
#Check_R
	la	$s3, Opcode_R_Check #chua list opcode mau
	li	$t1, 0		#i=0
Loop1_R:	
	add	$a3, $s3, $t1	#load byte cua opcode mau
	lb	$t3, 0($a3)
	addi	$t1, $t1, 1
	bne	$t3, $s7, Loop1_R 	#S7='/'
	li	$t0, 0		#So ki tu cua opcode mau
Loop2_R:
	add	$a3, $s3, $t1	#load byte cua opcode mau
	lb	$t3, 0($a3)
	add	$a2, $s2, $t0	#Load byte cua opcode can check
	lb	$t2, 0($a2)
	beq	$t3, $s7, Check_R	#'s7'='/'
	beq	$t3, $s6, End_Loop_R	#s6 = space
	bne	$t2, $t3, Loop1_R_them	#Kiem tra xem opcode check va opcode mau co giong nhau khong
	#khong giong nhay toi Loop1_R_them de bat dau ki tu tiep theo
	beq	$t2, $t3, Loop2_R_them	#Quay tro lai kiem tra next ki tu
End_Loop_R:

\end{lstlisting}
Đoạn mã này có chức năng kiểm tra xem opcode cần kiểm tra có khớp với opcode mẫu hay không. Cụ thể:
\begin{itemize}
    \item \textbf{Khởi tạo giá trị ban đầu:}
    \begin{itemize}
        \item \texttt{\$s3}: Địa chỉ chứa danh sách các opcode mẫu.
        \item \texttt{\$t1}: Biến đếm, khởi tạo bằng 0 (i=0).
    \end{itemize}

    \item \textbf{Vòng lặp Loop1\_R:}
    \begin{itemize}
        \item Lấy từng byte từ opcode mẫu và lưu vào \texttt{\$t3}.
        \item Tăng \texttt{\$t1} để dịch chuyển đến byte tiếp theo trong opcode mẫu.
        \item Nếu ký tự hiện tại là \texttt{'/'}, tiếp tục vòng lặp \texttt{Loop1\_R}.
    \end{itemize}

    \item \textbf{Khởi tạo giá trị \texttt{\$t0}:}
    \begin{itemize}
        \item Đặt \texttt{\$t0} là số ký tự của opcode mẫu (khởi tạo bằng 0).
    \end{itemize}

    \item \textbf{Vòng lặp Loop2\_R:}
    \begin{itemize}
        \item Lấy từng byte từ opcode mẫu và opcode cần kiểm tra, lưu vào \texttt{\$t3} và \texttt{\$t2} tương ứng.
        \item Nếu ký tự hiện tại trong opcode mẫu là \texttt{'/'}, nhảy đến \texttt{Check\_R}.
        \item Nếu ký tự hiện tại trong opcode mẫu là khoảng trắng (space), kết thúc vòng lặp \texttt{Loop2\_R}.
        \item So sánh ký tự hiện tại của opcode mẫu và opcode cần kiểm tra:
        \begin{itemize}
            \item Nếu không khớp, nhảy đến \texttt{Loop1\_R\_them} để bắt đầu lại vòng lặp \texttt{Loop1\_R} từ ký tự tiếp theo.
            \item Nếu khớp, tiếp tục vòng lặp \texttt{Loop2\_R\_them} để kiểm tra ký tự tiếp theo.
        \end{itemize}
    \end{itemize}

    \item \textbf{Kết thúc vòng lặp End\_Loop\_R:}
    \begin{itemize}
        \item Kết thúc quá trình kiểm tra và nhảy đến nhãn \texttt{End\_Loop\_R}.
    \end{itemize}
\end{itemize}

\textbf{Chức năng tổng quan:} Đoạn mã này kiểm tra từng ký tự của opcode cần kiểm tra với opcode mẫu để xác định xem chúng có khớp nhau hay không. Quá trình này lặp đi lặp lại cho đến khi toàn bộ các ký tự đã được so sánh hoặc phát hiện ký tự không khớp.

\subsection{Kiểm tra lệnh theo từng khuôn dạng lệnh}

Sau khi xác định khuôn dạng lệnh thông qua opcode, giá trị của thanh ghi \texttt{s4} sẽ được gán tương ứng với loại lệnh. Dựa trên giá trị của \texttt{s4}, chương trình sẽ nhảy đến các hàm kiểm tra tương ứng với từng khuôn dạng lệnh cụ thể.
\textbf{a. Ví dụ hàm kiểm tra khuôn lệnh R} (bao gồm các lệnh: \texttt{/add/sub/addu/subu/} \texttt{and/or/slt/sltu/} \texttt{nor/srav/srlv/} \texttt{movn/movz/mul}).
\begin{lstlisting} 
#----------------------------------------------------	
#Check cac thanh ghi va so
R_Check_Register_and_Number:
	jal	Right_opcode
	jal	Split_Register_and_Number
	jal	Check_Register
	jal	Split_Register_and_Number
	jal	Check_Register
	jal	Split_Register_and_Number
	jal	Check_Register
	addi	$t5, $zero, 10 		#s10= newline
	beq	$t9, $t5, Right_code
	addi	$t5, $zero, 0 		#t5=NULL
	beq	$t9, $t5, Right_code
	j	False_code
\end{lstlisting}
\begin{itemize}
    \item \textbf{Gọi hàm \texttt{Right\_opcode}}:
    \begin{itemize}
        \item Đầu tiên, hàm \texttt{Right\_opcode} được gọi để in ra thông báo rằng opcode trong câu lệnh là hợp lệ. Việc này đảm bảo rằng opcode đã được kiểm tra và xác nhận trước đó.
    \end{itemize}

    \item \textbf{Gọi hàm \texttt{Split\_Register\_and\_Number} và \texttt{Check\_Register}}:
    \begin{itemize}
        \item Tiếp theo, chúng ta cần kiểm tra 3 thanh ghi trong câu lệnh để đảm bảo chúng đúng.
        \item Đầu tiên, hàm \texttt{Split\_Register\_and\_Number} được gọi để tách thanh ghi ra khỏi chuỗi lệnh.
        \item Sau đó, hàm \texttt{Check\_Register} được gọi để kiểm tra xem thanh ghi đó có hợp lệ hay không.
        \item Vì câu lệnh dạng R có 3 thanh ghi, nên quá trình tách và kiểm tra này được lặp lại ba lần để kiểm tra cả ba thanh ghi.
    \end{itemize}

    \item \textbf{Kiểm tra ký tự cuối cùng (\texttt{\$t9})}:
    \begin{itemize}
        \item Sau khi kiểm tra xong các thanh ghi, chúng ta cần xác định xem ký tự cuối cùng của lệnh có hợp lệ không.
        \item Biến \texttt{\$t9} chứa ký tự cuối cùng của chuỗi lệnh.
        \item So sánh ký tự cuối cùng này với ký tự xuống dòng (newline, mã ASCII là 10) và ký tự NULL (mã ASCII là 0).
        \item Nếu ký tự cuối cùng là newline hoặc NULL, lệnh được xác nhận là đúng (\texttt{Right\_code}).
        \item Nếu không, lệnh được xác nhận là sai (\texttt{False\_code}).
    \end{itemize}
\end{itemize}
\textbf{b. Hàm kiểm tra khuôn lệnh I} (bao gồm các lệnh \texttt{/addi/addiu/andi/} \texttt{ori/slti/sltiu/sll/} \texttt{srl/sra/})
\begin{lstlisting}
I_Check_Register_and_Number:
	jal	Right_opcode
	jal	Split_Register_and_Number
	jal	Check_Register
	jal	Split_Register_and_Number
	jal	Check_Register
	jal	Split_Register_and_Number
	jal	Check_Number
	addi	$t5, $zero, 10
	beq	$t9, $t5, Right_code
	addi	$t5, $zero, 0
	beq	$t9, $t5, Right_code
	j	False_code
\end{lstlisting}
\textbf{Giải thích}
\begin{itemize}
    \item \textbf{Gọi hàm \texttt{Right\_opcode}}:
    \item \textbf{Kiểm tra hai thanh ghi đầu tiên}:
    \begin{itemize}
        \item Tiếp theo, chúng ta cần kiểm tra hai thanh ghi đầu tiên trong câu lệnh. Để làm điều này, chúng ta sử dụng hàm \texttt{Split\_Register\_and\_Number} để tách thanh ghi ra khỏi chuỗi lệnh.
        \item Sau đó, hàm \texttt{Check\_Register} được gọi để kiểm tra xem thanh ghi đó có hợp lệ hay không.
        \item Quá trình này được lặp lại hai lần để kiểm tra cả hai thanh ghi đầu tiên.
    \end{itemize}

    \item \textbf{Kiểm tra phần tử cuối cùng (số)}:
    \begin{itemize}
        \item Đối với phần tử cuối cùng của lệnh dạng I, chúng ta cần kiểm tra xem nó có phải là một số hợp lệ hay không.
        \item Hàm \texttt{Split\_Register\_and\_Number} được gọi để tách số ra khỏi chuỗi lệnh.
        \item Sau đó, hàm \texttt{Check\_Number} được gọi để kiểm tra xem số này có hợp lệ hay không.
    \end{itemize}

    \item \textbf{Kiểm tra ký tự cuối cùng (\texttt{\$t9})}:
    \begin{itemize}
      \item  Tương tự với hàm kiểm tra khuôn lệnh R
    \end{itemize}
\end{itemize}
\textbf{c. Hàm kiểm tra khuôn lệnh I đặc biệt}. (bao gồm 2 câu lệnh \texttt{beq, bne})
\begin{lstlisting}
    I_2_Check_Register_and_Number:
	jal	Right_opcode
	jal	Split_Register_and_Number
	jal	Check_Register
	jal	Split_Register_and_Number
	jal	Check_Register
	jal	Split_Register_and_Number
	addi	$t5, $zero, 10
	beq	$t9, $t5, R_1_Check_Label
	addi	$t5, $zero, 0
	beq	$t9, $t5, R_1_Check_Label
	j	False_code
	R_1_Check_Label:
	jal	Check_Label

\end{lstlisting}
\begin{itemize}
    \item \textbf{Gọi hàm \texttt{Right\_opcode}}
    \item \textbf{Kiểm tra hai thanh ghi đầu tiên:}
    Giống lệnh kiểm tra I bên trên.

    \item \textbf{Kiểm tra phần tử cuối cùng (nhãn)}:
    \begin{itemize}
        \item Sau khi kiểm tra xong các thanh ghi, ký tự cuối cùng của chuỗi lệnh (\texttt{\$t9}) được so sánh với ký tự xuống dòng (newline, mã ASCII là 10) và ký tự NULL (mã ASCII là 0).
        \item Nếu ký tự cuối cùng là newline hoặc NULL, chương trình sẽ nhảy đến nhãn \texttt{R\_1\_Check\_Label} để kiểm tra nhãn.
        \item Nếu không, lệnh được xác nhận là sai (\texttt{False\_code}).
    \end{itemize}

    \item \textbf{Kiểm tra nhãn (\texttt{Check\_Label})}:
    \begin{itemize}
        \item Tại nhãn \texttt{R\_1\_Check\_Label}, hàm \texttt{Check\_Label} được gọi để kiểm tra tính hợp lệ của nhãn trong lệnh.
    \end{itemize}
\end{itemize}

\textbf{d. Kiểm tra khuôn lệnh L} (bao gồm các lệnh \texttt{/lb/lbu/lh/lhu/ll/lw/sb/sc/....})
\begin{lstlisting}
    L_Check_Register_and_Number:
	jal	Right_opcode
	jal	Split_Register_and_Number
	jal	Check_Register
	jal	Check_Sign_ExtImm

\end{lstlisting}
\begin{itemize}
    \item \textbf{Gọi hàm \texttt{Right\_opcode}}

    \item \textbf{Tách và kiểm tra thanh ghi}:
    \begin{itemize}
        \item Hàm \texttt{Split\_Register\_and\_Number} được gọi để tách thanh ghi ra khỏi chuỗi lệnh.
        \item Sau đó, hàm \texttt{Check\_Register} được gọi để kiểm tra xem thanh ghi đó có hợp lệ hay không.
    \end{itemize}

    \item \textbf{Kiểm tra  (\texttt{Sign\_ExtImm})}:
    \begin{itemize}
        \item Cuối cùng, vì khuôn dạng như loại này thì nó có cái cụm phía sau khá đặc biệt nên ta sẽ xây dựng một hàm riêng để vừa cả tách đồng thời kiểm tra xem có thỏa mãn hay không.
    \end{itemize}
\end{itemize}

\subsection{Hàm tách thanh ghi và số}
\begin{lstlisting}
#----------------------------------------------------	
#Tach ma thanh ghi va so
Split_Register_and_Number:
	li	$s0, 0		#Vi tri phan tu cuoi cua mang chain_check
	add	$t1, $s5, $zero	#i=vi tri dang doc trong cau lenh=s5
Loop1_Split: #Bo qua space truoc thi ghi/so va load ki tu dau
	add	$a2, $s1, $t1	#a2 = Dia chi cua ky tu dang load
	add	$a3, $s2, $s0	#a3 = Dia chi dang nap vao hang doi
	lb	$t0, 0($a2)	#t0 = Ky tu dang Load
	add	$t9, $zero, $t0	#t9 = Ky tu cuoi cung duoc load
	beq	$t0, $zero, EndLoop_Split	#Gap null => ket thuc vong lap 1
	beq	$t0, $s6, Loop1_Split_them	#Gap Space -> Chay qua Space
	li	$t5, 44				#t5=44~'dau phay,' 
	beq	$t0, $t5, False_code
	sb	$t0, 0($a3) 		#Nap ky tu vao hang doi
	addi	$s0, $s0, 1		#Dich chuyen vi tri cuoi cua hang doi sang phai
	addi	$t1, $t1, 1
Loop2_Split: #load cac gia tri tiep theo vao hang doi
	add	$a2, $s1, $t1	#a2 = Dia chi cua ky tu dang load
	add	$a3, $s2, $s0	#a3 = Dia chi dang nap vao hang doi
	lb	$t0, 0($a2)
	add	$t9, $zero, $t0	#t9 = Ky tu cuoi cung duoc load
	beq	$t0, $zero, EndLoop_Split#Gap null => ket thuc vong lap 
	beq	$t0, $s6, Loop3_Split	#Gap space => Chay qua Space
	li	$t5, 10			#t5=newline
	beq	$t0, $t5, EndLoop_Split	#Gap newline => ket thuc vong lap 
	li	$t5, 44			#t5=44~'dau phay,' 
	beq	$t0, $t5, EndLoop_Split	#Gap dau phay => ket thuc vong lap 
	sb	$t0, 0($a3) 		#Nap ky tu vao hang doi
	addi	$s0, $s0, 1		#Dich chuyen vi tri cuoi cua hang doi sang phai
	addi	$t1, $t1, 1
	j	Loop2_Split
Loop3_Split: #Loai bo cac ki tu khoang trang dung sau thanh ghi va so
	add	$a2, $s1, $t1	#a2 = Dia chi cua ky tu dang load
	add	$a3, $s2, $s0	#a3 = Dia chi dang nap vao hang doi
	lb	$t0, 0($a2)	#t0 = Ky tu dang Load
	add	$t9, $zero, $t0	#t9 = Ky tu cuoi cung duoc load
	beq	$t0, $zero, EndLoop_Split	#Gap null => ket thuc vong lap 1
	beq	$t0, $s6, Loop3_Split_them	#Gap Space -> Chay qua Space
	li	$t5, 44		#t5=44~'dau phay,' 
	beq	$t0, $t5, EndLoop_Split
	li	$t5, 10		#t5=10~'New line' 
	beq	$t0, $t5, EndLoop_Split
	j	False_code
EndLoop_Split: #hoan thanh viec tach
	#Chen ky tu NULL cho hang doi
	sb	$zero, 0($a3)
	addi	$s5, $t1, 1	#Luu vi tri ki tu dang doc vao s5
	addi	$s0, $s0, -1
	jr	$ra
\end{lstlisting}
\textbf{Giải thích}
\begin{itemize}
    \item \textbf{Loop1\_Split:}
    \begin{itemize}
        \item \textit{Mục đích:} Bỏ qua các ký tự khoảng trắng (space) đứng trước thanh ghi hoặc số và load ký tự đầu vào hàng đợi.
        \item \textit{Cách thực hiện:}
        \begin{itemize}
            \item Load ký tự từ địa chỉ \texttt{\$s1} với vị trí \texttt{\$t1}.
            \item Nếu ký tự là \texttt{null}, kết thúc vòng lặp.
            \item Nếu ký tự là \texttt{space}, bỏ qua và tiếp tục vòng lặp.
            \item Nếu ký tự là dấu phẩy (\texttt{,}), chuyển đến hàm \texttt{False\_code}.
            \item Lưu ký tự vào hàng đợi và di chuyển đến ký tự tiếp theo.
        \end{itemize}
    \end{itemize}

    \item \textbf{Loop2\_Split:}
    \begin{itemize}
        \item \textit{Mục đích:} Load các ký tự tiếp theo vào hàng đợi.
        \item \textit{Cách thực hiện:}
        \begin{itemize}
            \item Load ký tự từ địa chỉ \texttt{\$s1} với vị trí \texttt{\$t1}.
            \item Nếu ký tự là \texttt{null}, kết thúc vòng lặp.
            \item Nếu ký tự là \texttt{space}, chuyển sang \texttt{Loop3\_Split}.
            \item Nếu ký tự là \texttt{newline}, kết thúc vòng lặp.
            \item Nếu ký tự là dấu phẩy (\texttt{,}), kết thúc vòng lặp.
            \item Lưu ký tự vào hàng đợi và di chuyển đến ký tự tiếp theo.
        \end{itemize}
    \end{itemize}

    \item \textbf{Loop3\_Split:}
    \begin{itemize}
        \item \textit{Mục đích:} Loại bỏ các ký tự khoảng trắng đứng sau thanh ghi hoặc số.
        \item \textit{Cách thực hiện:}
        \begin{itemize}
            \item Load ký tự từ địa chỉ \texttt{\$s1} với vị trí \texttt{\$t1}.
            \item Nếu ký tự là \texttt{null}, kết thúc vòng lặp.
            \item Nếu ký tự là \texttt{space}, bỏ qua và tiếp tục vòng lặp.
            \item Nếu ký tự là dấu phẩy (\texttt{,}), kết thúc vòng lặp.
            \item Nếu ký tự là \texttt{newline}, kết thúc vòng lặp.
            \item Nếu ký tự không hợp lệ, chuyển đến hàm \texttt{False\_code}.
        \end{itemize}
    \end{itemize}

    \item \textbf{EndLoop\_Split:}
    \begin{itemize}
        \item \textit{Mục đích:} Hoàn tất quá trình tách thanh ghi và số.
        \item \textit{Cách thực hiện:}
        \begin{itemize}
            \item Chèn ký tự \texttt{NULL} vào cuối hàng đợi để đánh dấu kết thúc chuỗi.
            \item Cập nhật giá trị \texttt{\$s5} và \texttt{\$s0}.
            \item Quay về chương trình chính.
        \end{itemize}
    \end{itemize}
\end{itemize}
\subsection{Hàm kiểm tra thanh ghi}
\begin{lstlisting}
    #----------------------------------------------------	
#----------------------------------------------------	
#Check Register
Check_Register:
	la	$s3, Register_Check	#dia chi cua list thanh ghi mau
	li	$t1, 0			#i=0
Loop1_Reg: #Duyet qua cac ki tu cua thanh ghi mau va bo qua ki tu '/'
	add	$a3, $s3, $t1	#load byte cua thanh ghi mau
	lb	$t3, 0($a3)
	addi	$t1, $t1, 1
	bne	$t3, $s7, Loop1_Reg	#s7=/
	li	$t0, 0		#So ki tu cua thanh ghi mau
Loop2_Reg: #so sanh thanh ghi mau va thanh ghi kiem tra.
	add	$a3, $s3, $t1	#load byte cua thanh ghi mau
	lb	$t3, 0($a3)
	add	$a2, $s2, $t0	#Load byte cua thanh ghi can check
	lb	$t2, 0($a2)
	beq	$t3, $s7, Check_Reg
	beq	$t3, $s6, False_code
	bne	$t2, $t3, Loop1_Reg_them	#Kiem tra xem thanh ghi check va thanh ghi mau co giong nhau khong
	beq	$t2, $t3, Loop2_Reg_them	
End_Loop_Reg:

#----------------------------------------------------

Check_Reg: #kiem tra ki tu cuoi cung cua Reg co khop voi thanh ghi mau khong?
	addi	$t0, $t0, -1
	beq	$s0, $t0, Reg_True
	j	Loop1_Reg
Loop1_Reg_them:
	addi	$t1, $t1, 1
	j	Loop1_Reg
Loop2_Reg_them:
	addi	$t1, $t1, 1
	addi	$t0, $t0, 1
	j	Loop2_Reg
Reg_True:
	add 	$t8, $zero, $ra
	jal	Right_Register
	jr	$t8

\end{lstlisting}
\textbf{a. Giải thích chức năng của các thanh ghi}
\begin{table}[H]
\centering
\begin{tabular}{|c|l|}
\hline
\textbf{Thanh ghi} & \textbf{Ý nghĩa} \\
\hline
\texttt{\$s2} & Địa chỉ của ký tự đầu tiên trong hàng đợi \\
\texttt{\$s3} & Địa chỉ của chuỗi thanh ghi mẫu đã lưu trữ trước đó \\
\texttt{\$s0} & Vị trí cuối cùng trong hàng đợi \\
\texttt{\$a2} & Địa chỉ của ký tự thanh ghi đang được nạp từ hàng đợi để kiểm tra \\
\texttt{\$a3} & Địa chỉ của ký tự đang được nạp từ chuỗi thanh ghi mẫu \\
\texttt{\$t1} & Biến đếm \texttt{i}, khởi tạo bằng 0 \\
\texttt{\$t2} & Ký tự được nạp từ \texttt{\$a2} \\
\texttt{\$t3} & Ký tự được nạp từ \texttt{\$a3} \\
\texttt{\$t0} & Số lượng ký tự của chuỗi thanh ghi mẫu \\
\hline
\end{tabular}
\caption{Chức năng các thanh ghi trong hàm \texttt{Check\_Register}}
\label{tab:register_functions}
\end{table}

\textbf{b. Giải thích code}
\begin{itemize}
    \item \textbf{Khởi tạo giá trị ban đầu:}
    \begin{itemize}
        \item \texttt{\$s3}: Địa chỉ chứa danh sách các thanh ghi hợp lệ (\texttt{Register\_Check}).
        \item \texttt{\$t1}: Biến đếm, khởi tạo bằng 0 (\texttt{i=0}).
    \end{itemize}

    \item \textbf{Vòng lặp \texttt{Loop1\_Reg}:}
    \begin{itemize}
        \item Tải byte của thanh ghi mẫu từ địa chỉ \texttt{\$s3} và lưu vào \texttt{\$t3}.
        \item Tăng biến đếm \texttt{\$t1} lên 1.
        \item Nếu ký tự \texttt{\$t3} là ký tự '/', tiếp tục vòng lặp \texttt{Loop1\_Reg}.
        \item Đặt \texttt{\$t0} là số ký tự của thanh ghi mẫu.
    \end{itemize}
\textbf{ => Dùng để duyệt qua các ký tự của thanh ghi mẫu và bỏ qua các ký tự '/'.}
    \item \textbf{Vòng lặp \texttt{Loop2\_Reg}:}
    \begin{itemize}
        \item Tải byte của thanh ghi mẫu từ địa chỉ \texttt{\$s3} và lưu vào \texttt{\$t3}.
        \item Tải byte của thanh ghi cần kiểm tra từ hàng đợi (\texttt{\$s2}) và lưu vào \texttt{\$t2}.
        \item Nếu ký tự \texttt{\$t3} là ký tự '/', nhảy đến \texttt{Check\_Reg}.
        \item Nếu ký tự \texttt{\$t3} là ký tự ' ', nhảy đến \texttt{False\_code}.
        \item Nếu ký tự \texttt{\$t2} và \texttt{\$t3} không khớp, nhảy đến \texttt{Loop1\_Reg\_them}.
        \item Nếu ký tự \texttt{\$t2} và \texttt{\$t3} khớp, tiếp tục vòng lặp \texttt{Loop2\_Reg\_them}.
    \end{itemize}
\textbf{=> Dùng để kiểm tra xem thanh ghi cần kiểm tra có khớp với thanh ghi mẫu hay không. Nếu không khớp, nó sẽ nhảy đến 
\texttt{Loop1\_Reg\_them hoặc False\_code.}}

    \item \textbf{Vòng lặp \texttt{Loop1\_Reg\_them}:}
    \begin{itemize}
        \item Tăng biến đếm \texttt{\$t1} lên 1.
        \item Quay lại \texttt{Loop1\_Reg}.
    \end{itemize}
    \textbf{=> Dùng để tăng biến đếm \texttt{\$t1} và quay lại \texttt{Loop1\_Reg}}.
    \item \textbf{Vòng lặp \texttt{Loop2\_Reg\_them}:}
    \begin{itemize}
        \item Tăng biến đếm \texttt{\$t1} lên 1.
        \item Tăng \texttt{\$t0} lên 1.
        \item Quay lại \texttt{Loop2\_Reg}.
    \end{itemize}
\textbf{=> Dùng để tăng biến đếm \texttt{\$t1}, \texttt{\$t0} và quay lại \texttt{Loop2\_Reg}.}
    \item \textbf{Hàm \texttt{Check\_Reg}:}
    \begin{itemize}
        \item Giảm \texttt{\$t0} đi 1.
        \item Nếu \texttt{\$s0} bằng \texttt{\$t0}, nhảy đến \texttt{Reg\_True}.
        \item Quay lại \texttt{Loop1\_Reg}.
    \end{itemize}
\textbf{=> Dùng để kiểm tra xem ký tự cuối cùng của thanh ghi cần kiểm tra có khớp với thanh ghi mẫu không. Nếu khớp, nhảy đến \texttt{Reg\_True}.}
    \item \textbf{Nhãn \texttt{Reg\_True}:}
    \begin{itemize}
        \item Đặt \texttt{\$t8} bằng giá trị trả về của thanh ghi ra.
        \item Gọi hàm \texttt{Right\_Register} để in ra thông báo rằng thanh ghi hợp lệ.
        \item Quay lại địa chỉ trong \texttt{\$t8}.
    \end{itemize}
\textbf{=> Xác nhận thanh ghi hợp lệ và gọi hàm \texttt{Right\_Register} để in thông báo}
\end{itemize}

\subsection {Kiểm tra số}
\begin{lstlisting}
#----------------------------------------------------	
#Check Number
Check_Number:
	li	$t1, 0		#i = 0
	j	Check_Mark
Check_Mark_done: #Kiem tra cac ki tu tiep theo
	add	$a2, $s2, $t1	#Kiem tra so dau tien
	lb	$t2, 0($a2)
	li	$t5, 10		#t5 = newline
	beq	$t2, $t5, False_code
	beq	$t2, $zero, False_code
	li	$t5, 48		#t5 = zero
	bne	$t2, $t5, Loop_Number_1
	slti	$t4, $t2, 48
	bne	$t4, $zero, False_code
	slti	$t4, $t2, 58 
	beq	$t4, $zero, False_code
	addi	$t1, $t1, 1	#Kiem tra so thu hai(co the la x trong so hexa)
	add	$a2, $s2, $t1
	lb	$t2, 0($a2)
	#Kiem tra xem cac ki tu co nam trong khoang so hop le khong (0-9, A-F, a-f)
	beq	$t2, $zero, Right_Number
	li	$t5, 120
	beq	$t2, $t5, Loop_Number_them
	li	$t5, 88
	beq	$t2, $t5, Loop_Number_them
	slti	$t4, $t2, 48
	bne	$t4, $zero, False_code
	slti	$t4, $t2, 58 
	beq	$t4, $zero, False_code
Loop_Number: #kiem tra cac ki tu so
	add	$a2, $s2, $t1
	lb	$t2, 0($a2)
	beq	$t2, $zero, Right_Number
	li	$t5, 48
	beq	$t2, $t5, Loop_Number_them
	li	$t5, 49
	beq	$t2, $t5, Loop_Number_them
	li	$t5, 50
	beq	$t2, $t5, Loop_Number_them
	li	$t5, 51
	beq	$t2, $t5, Loop_Number_them
	li	$t5, 52
	beq	$t2, $t5, Loop_Number_them
	li	$t5, 53
	beq	$t2, $t5, Loop_Number_them
	li	$t5, 54
	beq	$t2, $t5, Loop_Number_them
	li	$t5, 55
	beq	$t2, $t5, Loop_Number_them
	li	$t5, 56
	beq	$t2, $t5, Loop_Number_them
	li	$t5, 57
	beq	$t2, $t5, Loop_Number_them
	li	$t5, 65
	beq	$t2, $t5, Loop_Number_them
	li	$t5, 66
	beq	$t2, $t5, Loop_Number_them
	li	$t5, 67
	beq	$t2, $t5, Loop_Number_them
	li	$t5, 68
	beq	$t2, $t5, Loop_Number_them
	li	$t5, 69
	beq	$t2, $t5, Loop_Number_them
	li	$t5, 70
	beq	$t2, $t5, Loop_Number_them
	li	$t5, 97
	beq	$t2, $t5, Loop_Number_them
	li	$t5, 98
	beq	$t2, $t5, Loop_Number_them
	li	$t5, 99
	beq	$t2, $t5, Loop_Number_them
	li	$t5, 100
	beq	$t2, $t5, Loop_Number_them
	li	$t5, 101
	beq	$t2, $t5, Loop_Number_them
	li	$t5, 102
	beq	$t2, $t5, Loop_Number_them
	j	False_code
Loop_Number_1:
	add	$a2, $s2, $t1
	lb	$t2, 0($a2)
	beq	$t2, $zero, Right_Number
	li	$t5, 48
	beq	$t2, $t5, Loop_Number_them_1
	li	$t5, 49
	beq	$t2, $t5, Loop_Number_them_1
	li	$t5, 50
	beq	$t2, $t5, Loop_Number_them_1
	li	$t5, 51
	beq	$t2, $t5, Loop_Number_them_1
	li	$t5, 52
	beq	$t2, $t5, Loop_Number_them_1
	li	$t5, 53
	beq	$t2, $t5, Loop_Number_them_1
	li	$t5, 54
	beq	$t2, $t5, Loop_Number_them_1
	li	$t5, 55
	beq	$t2, $t5, Loop_Number_them_1
	li	$t5, 56
	beq	$t2, $t5, Loop_Number_them_1
	li	$t5, 57
	j	False_code
#----------------------------------------------------
Right_Number:
	add	$t8, $zero, $ra
	jal 	Print_Right_Number
	jr	$t8
#----------------------------------------------------	
Check_Mark:	#Ham kiem tra dau cua imm
	add	$a2, $s2, $t1	#Kiem tra xem ki tu dau tien cua Imm co phai dau + hay - khong?
	lb	$t2, 0($a2)
	li	$t5, 43		#t5 =43 ~ '+'
	beq	$t2, $t5, Check_Mark_them
	li	$t5, 45		#t5 =45 ~ '-'
	beq	$t2, $t5, Check_Mark_them
	j	Check_Mark_done
\end{lstlisting}
\textbf{a. Giải thích ý nghĩa của thanh ghi}
\begin{table}[H]
\centering
\begin{tabular}{|l|p{10cm}|}
\hline
\texttt{\$s2} & Địa chỉ ký tự đầu của hàng đợi \\ \hline
\texttt{\$s0} & Vị trí cuối của hàng đợi \\ \hline
\texttt{\$a2} & Địa chỉ ký tự thanh ghi đang load từ hàng đợi đang cần kiểm tra \\ \hline
\texttt{\$t1} & Biến đếm, mặc định ban đầu = 0 \\ \hline
\texttt{\$t2} & Ký tự load từ \texttt{\$a2} \\ \hline
\texttt{\$t3} & Ký tự load từ \texttt{\$a3} \\ \hline
\texttt{\$t5} & Ký tự tạm thời dùng để kiểm tra \\ \hline
\end{tabular}
\caption{Giải thích chức năng các thanh ghi}
\end{table}
\textbf{b. Giải thích chương trình}
\begin{itemize}
    \item \textbf{Khởi tạo biến đếm:}
    \begin{itemize}
        \item Đặt biến đếm \texttt{\$t1} về 0 và nhảy đến \texttt{Check\_Mark} để kiểm tra dấu của số.
    \end{itemize}

    \item \textbf{Kiểm tra dấu số:}
    \begin{itemize}
        \item Kiểm tra xem ký tự đầu tiên có phải là dấu cộng hoặc trừ không. Nếu đúng, nhảy tới \texttt{Check\_Mark\_them} để bỏ qua dấu này và tiếp tục kiểm tra các ký tự tiếp theo. Nếu không phải, nhảy đến \texttt{Check\_Mark\_done}.
    \end{itemize}

    \item \textbf{Kiểm tra các ký tự tiếp theo:}
    \begin{itemize}
        \item Kiểm tra ký tự đầu tiên sau dấu (nếu có). Nếu gặp ký tự \texttt{newline} hoặc \texttt{NULL} thì nhảy đến \texttt{False\_code}.
        \item Nếu ký tự đầu tiên không phải là '0', chuyển sang \texttt{Loop\_Number}.
        \item Nếu ký tự đầu tiên là '0', kiểm tra ký tự tiếp theo có phải là 'x' (ký hiệu số hexa) không.
        \item Kiểm tra các ký tự tiếp theo xem có nằm trong khoảng số hợp lệ (0-9, A-F, a-f).
    \end{itemize}

    \item \textbf{Vòng lặp kiểm tra các ký tự số:}
    \begin{itemize}
        \item Vòng lặp \texttt{Loop\_Number} kiểm tra từng ký tự trong chuỗi xem có phải là ký tự số hợp lệ hay không.
        \item Nếu ký tự là số (0-9) hoặc chữ cái trong dải hợp lệ cho số hexa (A-F, a-f), chuyển đến \texttt{Loop\_Number\_them} để tiếp tục kiểm tra ký tự tiếp theo.
        \item Nếu gặp ký tự không hợp lệ, nhảy đến \texttt{False\_code}.
    \end{itemize}

    \item \textbf{Kiểm tra ký tự tiếp theo trong số:}
    \begin{itemize}
        \item Tăng biến đếm \texttt{\$t1} và quay lại vòng lặp \texttt{Loop\_Number} để kiểm tra ký tự tiếp theo.
    \end{itemize}

    \item \textbf{Xác nhận số hợp lệ:}
    \begin{itemize}
        \item Nếu tất cả các ký tự trong chuỗi đều hợp lệ, hàm \texttt{Right\_Number} được gọi để xác nhận số hợp lệ và quay lại địa chỉ trước đó.
    \end{itemize}
\end{itemize}
\subsection{Kiểm tra nhãn Label}
\begin{lstlisting}
#----------------------------------------------------	
#Check Label
Check_Label:
	li	$t1, 0		#i = 0
	add	$a2, $s2, $t1
	lb	$t2, 0($a2)
	beq	$t2, $zero, False_code
	li	$t5, 10		#t5 = 'New line'
	beq	$t2, $t5, False_code
	slti	$t4, $t2, 48
	bne	$t4, $zero, False_code
	li	$t5, 58
	beq	$t2, $t5, False_code
	li	$t5, 59
	beq	$t2, $t5, False_code
	li	$t5, 60
	beq	$t2, $t5, False_code
	li	$t5, 61
	beq	$t2, $t5, False_code
	li	$t5, 62
	beq	$t2, $t5, False_code
	li	$t5, 63
	beq	$t2, $t5, False_code
	li	$t5, 64
	beq	$t2, $t5, False_code
	li	$t5, 91
	beq	$t2, $t5, False_code
	li	$t5, 92
	beq	$t2, $t5, False_code
	li	$t5, 93
	beq	$t2, $t5, False_code
	li	$t5, 94
	beq	$t2, $t5, False_code
	li	$t5, 96
	beq	$t2, $t5, False_code
	slti	$t4, $t2, 123
	beq	$t4, $zero, False_code
	addi	$t1, $t1, 1
Loop_Label:
	add $a2, $s2, $t1  		# Lay dia chi cua ky tu tiep theo tu hang doi
	lb $t2, 0($a2)    		# Lay ky tu tiep theo va luu vao $t2
	beq $t2, $zero, True_Label     	# Neu gap ky tu NULL, nhay den True_Label
	li $t5, 10                     	# t5 = 'New line'
	beq $t2, $t5, True_Label	# Neu gap ky tu newline, nhay den True_Label
	
	slti $t4, $t2, 48
	bne $t4, $zero, False_code       # Neu ky tu nho hon '0' (ma ASCII 48), nhay den False_code
	li $t5, 58
	beq $t2, $t5, False_code         # Neu ky tu la ':', nhay den False_code
	li $t5, 59
	beq $t2, $t5, False_code         # Neu ky tu la ';', nhay den False_code
	li $t5, 60
	beq $t2, $t5, False_code         # Neu ky tu la '<', nhay den False_code
	li $t5, 61
	beq $t2, $t5, False_code         # Neu ky tu la '=', nhay den False_code
	li $t5, 62
	beq $t2, $t5, False_code         # Neu ky tu la '>', nhay den False_code
	li $t5, 63
	beq $t2, $t5, False_code         # Neu ky tu la '?', nhay den False_code
	li $t5, 64
	beq $t2, $t5, False_code         # Neu ky tu la '@', nhay den False_code
	li $t5, 91
	beq $t2, $t5, False_code         # Neu ky tu la '[', nhay den False_code
	li $t5, 92
	beq $t2, $t5, False_code         # Neu ky tu la '\', nhay den False_code
	li $t5, 93
	beq $t2, $t5, False_code         # Neu ky tu la ']', nhay den False_code
	li $t5, 94
	beq $t2, $t5, False_code         # Neu ky tu la '^', nhay den False_code
	li $t5, 96
	beq $t2, $t5, False_code         # Neu ky tu la '`', nhay den False_code
	slti $t4, $t2, 123
	beq $t4, $zero, False_code       # Neu ky tu lon hon 'z' (ma ASCII 122), nhay den False_code

	
	addi	$t1, $t1, 1
	j	Loop_Label
	
#----------------------------------------------------
True_Label:
	jal 	Print_Right_Label
	j	Right_code

\end{lstlisting}
\textbf{Giải thích}
\textbf{Giải thích}
\begin{itemize}
    \item \textbf{Check\_Label}
    \begin{itemize}
        \item Khởi tạo giá trị ban đầu:
        \begin{itemize}
            \item \$t1 được đặt về 0 để bắt đầu đếm từ ký tự đầu tiên của chuỗi cần kiểm tra. 
            \item Lấy ký tự đầu tiên từ địa chỉ \texttt{\$s2} (hàng đợi) và lưu vào \texttt{\$t2} để kiểm tra.
        \end{itemize}
        \item Kiểm tra ký tự đầu tiên:
        \begin{itemize}
            \item Nếu ký tự đầu tiên là \texttt{NULL} hoặc \texttt{newline}, chương trình kết thúc với \texttt{False\_code} vì nhãn không hợp lệ.
            \item Kiểm tra ký tự đầu tiên có nằm trong dải ký tự số hoặc ký tự hợp lệ cho nhãn hay không. Nếu không hợp lệ, nhảy đến \texttt{False\_code}.
        \end{itemize}
    \end{itemize}
    \item \textbf{Loop\_Label}
    \begin{itemize}
        \item Vòng lặp này tiếp tục kiểm tra các ký tự tiếp theo trong chuỗi.
        \item Lấy ký tự tiếp theo từ hàng đợi và lưu vào \texttt{\$t2}.
        \item Nếu ký tự là \texttt{NULL} hoặc \texttt{newline}, nhảy đến \texttt{True\_Label}.
        \item Kiểm tra ký tự có nằm trong dải ký tự số hoặc ký tự hợp lệ cho nhãn hay không. Nếu không hợp lệ, nhảy đến \texttt{False\_code}.
        \item Tăng biến đếm \texttt{\$t1} và tiếp tục vòng lặp để kiểm tra ký tự tiếp theo.
    \end{itemize}
    \item \textbf{True\_Label}
    \begin{itemize}
        \item Khi tất cả các ký tự của nhãn đều hợp lệ, hàm \texttt{Print\_Right\_Label} được gọi để in thông báo nhãn hợp lệ.
        \item Chương trình tiếp tục với nhãn \texttt{Right\_code}.
    \end{itemize}
\end{itemize}

\subsection{Tách và kiểm tra cấu trúc đặc biệt (lw,sw,lb,sb,lh,sh)}
\subsubsection{a. Tách  }
\begin{lstlisting}
    #----------------------------------------------------	
#Tach Sign ExtImm
Split_Sign_ExtImm:
	li	$s0, 0		#Vi tri phan tu cuoi cua mang chain_check
	add	$t1, $s5, $zero	#i=vi tri dang doc trong cau lenh=s5
Loop1_Sign:
	add	$a2, $s1, $t1	#a2 = Dia chi cua ky tu dang load
	add	$a3, $s2, $s0	#a3 = Dia chi dang nap vao hang doi
	lb	$t0, 0($a2)	#t0 = Ky tu dang Load
	add	$t9, $zero, $t0	#t9 = Ky tu cuoi cung duoc load
	beq	$t0, $zero, EndLoop_Sign_them_2#Check_Reg_and_Num	#Gap null => ket thuc vong lap 1
	li	$t5, 10		#t5=10~'New line' 
	beq	$t0, $t5, EndLoop_Sign_them_2 
	beq	$t0, $s6, Loop1_Sign_them	#Gap Space -> Chay qua Space
	li	$t5, 44		#t5=44~'dau phay,' 
	beq	$t0, $t5, False_code
	sb	$t0, 0($a3) 		#Nap ky tu vao hang doi
	li	$t5, 40			#Thay dau ( thi ket thuc
	beq	$t0, $t5, EndLoop_Sign_them
	li	$t5, 41			#Thay dau ) thi ket thuc
	beq	$t0, $t5, EndLoop_Sign_them_3
	addi	$s0, $s0, 1		#Dich chuyen vi tri cuoi cua hang doi sang phai
	addi	$t1, $t1, 1
Loop2_Sign:
	add	$a2, $s1, $t1	#a2 = Dia chi cua ky tu dang load
	add	$a3, $s2, $s0	#a3 = Dia chi dang nap vao hang doi
	lb	$t0, 0($a2)
	add	$t9, $zero, $t0	#t9 = Ky tu cuoi cung duoc load
	beq	$t0, $zero, EndLoop_Sign_them_2#Check_Reg_and_Num	#Gap null => ket thuc vong lap 1
	li	$t5, 10		#t5=10~'New line' 
	beq	$t0, $t5, EndLoop_Sign_them_2 
	beq	$t0, $s6, EndLoop_Sign	#Gap space => Chay qua Space
	li	$t5, 10			#t5=newline
	beq	$t0, $t5, EndLoop_Sign	#Check_Reg_and_Num	#Gap newline => ket thuc vong lap 1
	li	$t5, 44			#t5=44~'dau phay,' 
	beq	$t0, $t5, EndLoop_Sign	#Gap dau phay => ket thuc vong lap 1
	li	$t5, 40			#Thay dau ( thi ket thuc
	beq	$t0, $t5, EndLoop_Sign_them_1
	li	$t5, 41			#Thay dau ) thi ket thuc
	beq	$t0, $t5, EndLoop_Sign_them_1
	sb	$t0, 0($a3) 		#Nap ky tu vao hang doi
	addi	$s0, $s0, 1		#Dich chuyen vi tri cuoi cua hang doi sang phai
	addi	$t1, $t1, 1
	j	Loop2_Sign
EndLoop_Sign:
	#Chen ky tu NULL cho hang doi
	sb	$zero, 0($a3)
	addi	$s5, $t1, 0
	addi	$s0, $s0, -1
	jr	$ra

#----------------------------------------------------
\end{lstlisting}
\textbf{Giải thích}
\begin{itemize}
    \item \textbf{Khởi tạo giá trị ban đầu}
    \begin{itemize}
    \item \texttt{\$s0}: Đặt về 0, vị trí phần tử cuối của hàng đợi.
    \item \texttt{\$t1}: Đặt bằng giá trị của \texttt{\$s5}, vị trí đang đọc trong chuỗi lệnh.
\end{itemize} \textbf{Vòng lặp Loop1\_Sign}
    \begin{itemize}
    \item Lấy địa chỉ của ký tự đang load từ chuỗi lệnh vào \texttt{\$a2}.
    \item Lấy địa chỉ của ký tự đang load từ hàng đợi vào \texttt{\$a3}.
    \item Tải ký tự từ chuỗi lệnh vào \texttt{\$t0}.
    \item Lưu ký tự cuối cùng được load vào \texttt{\$t9}.
    \item Kiểm tra ký tự:
    \begin{itemize}
        \item Nếu là ký tự \texttt{NULL}, nhảy đến \texttt{EndLoop\_Sign\_them\_2}.
        \item Nếu là ký tự \texttt{newline}, nhảy đến \texttt{EndLoop\_Sign\_them\_2}.
        \item Nếu là ký tự \texttt{space}, bỏ qua và tiếp tục vòng lặp.
        \item Nếu là dấu phẩy, nhảy đến \texttt{False\_code}.
    \end{itemize}
    \item Lưu ký tự vào hàng đợi.
    \item Kiểm tra các ký tự đặc biệt:
    \begin{itemize}
        \item Nếu là dấu \texttt{(}, nhảy đến \texttt{EndLoop\_Sign\_them}.
        \item Nếu là dấu \texttt{)}, nhảy đến \texttt{EndLoop\_Sign\_them\_3}.
    \end{itemize}
    \item Cập nhật vị trí cuối của hàng đợi và tiếp tục vòng lặp.
\end{itemize}
    \item \textbf{Vòng lặp Loop2\_Sign}
    \begin{itemize}
    \item Tương tự như \texttt{Loop1\_Sign}, tiếp tục tải và xử lý các ký tự.
    \item Kiểm tra ký tự:
    \begin{itemize}
        \item Nếu là ký tự \texttt{NULL}, nhảy đến \texttt{EndLoop\_Sign\_them\_2}.
        \item Nếu là ký tự \texttt{newline}, nhảy đến \texttt{EndLoop\_Sign\_them\_2}.
        \item Nếu là ký tự \texttt{space}, kết thúc vòng lặp.
        \item Nếu là ký tự \texttt{newline}, kết thúc vòng lặp.
        \item Nếu là dấu phẩy, kết thúc vòng lặp.
        \item Nếu là dấu \texttt{(}, nhảy đến \texttt{EndLoop\_Sign\_them\_1}.
        \item Nếu là dấu \texttt{)}, nhảy đến \texttt{EndLoop\_Sign\_them\_1}.
    \end{itemize}
    \item Lưu ký tự vào hàng đợi.
    \item Cập nhật vị trí cuối của hàng đợi và tiếp tục vòng lặp.
\end{itemize}
\item \textbf{Kết thúc vòng lặp}
\begin{itemize}
    \item \texttt{EndLoop\_Sign}: Thêm ký tự \texttt{NULL} vào cuối hàng đợi để đánh dấu kết thúc chuỗi.
    \item Cập nhật giá trị \texttt{\$s5}: Lưu vị trí ký tự đang đọc vào \texttt{\$s5}.
    \item Cập nhật vị trí cuối của hàng đợi: Giảm \texttt{\$s0} đi 1.
    \item Quay trở lại hàm chính.
\end{itemize}
\end{itemize}

\subsubsection{b. Check Sign\_ExtImm}
\begin{lstlisting}
   #----------------------------------------------------	
#Check Sign_ExtImm
Check_Sign_ExtImm:
	add	$t8, $zero, $ra		#Luu dia chi tro ve chuong trinh vao -> t8
	jal	Split_Sign_ExtImm	#tach va luu gia tri cua imm vao queue
	jal	Check_Number		#kiem tra xem gia tri imm hop le khong
	jal	Split_Sign_ExtImm	#tiep tuc xu li phan con lai cua chuoi lenh
	jal	Check_Parentheses_1	#kiem tra ki tu '('
	jal	Split_Sign_ExtImm	#tiep tuc xu li phan con lai cua chuoi lenh
	jal	Check_Register		#Kiem tra xem thanh ghi co hop le khong
	jal	Split_Sign_ExtImm	#tiep tuc xu li phan con lai cua chuoi lenh
	jal	Check_Parentheses_2	#xu li ky tu ')'
	addi	$t5, $zero, 10		#10: newline
	beq	$t9, $t5, Right_code
	addi	$t5, $zero, 0
	beq	$t9, $t5, Right_code
	addi	$t5, $zero, 41		#t5 ~ ')'
	beq	$t9, $t5, Right_code
	j	False_code
	
#Check_Parentheses_1   Kiem tra dau (
Check_Parentheses_1:
	li	$t1, 0	#i = 0
	add	$a2, $s2, $t1
	lb	$t2, 0($a2)
	li	$t5, 40
	bne	$t2, $t5, False_code
	addi	$t1, $t1, 1
	add	$a2, $s2, $t1
	lb	$t2, 0($a2)
	bne	$zero, $t2, False_code
	jr	$ra

#Check_Parentheses_2   Kiem tra dau )
Check_Parentheses_2:
	li	$t1, 0	#i = 0
	add	$a2, $s2, $t1
	lb	$t2, 0($a2)
	li	$t5, 41
	bne	$t2, $t5, False_code
	addi	$t1, $t1, 1
	add	$a2, $s2, $t1
	lb	$t2, 0($a2)
	bne	$zero, $t2, False_code
	jr	$ra



\end{lstlisting}
\textbf{Giải thích hàm Check\_Sign\_ExtImm}
\begin{itemize}
    \item \textbf{Lưu địa chỉ trở về:}
    \begin{itemize}
        \item Giá trị của thanh ghi \texttt{\$ra} (địa chỉ trở về) được lưu vào thanh ghi tạm thời \texttt{\$t8}.
    \end{itemize}

    \item \textbf{Gọi các hàm phụ trợ:}
    \begin{itemize}
        \item Gọi hàm \texttt{Split\_Sign\_ExtImm} để tách và lưu giá trị của Immediate (số mở rộng dấu) vào hàng đợi.
        \item Gọi hàm \texttt{Check\_Number} để kiểm tra xem giá trị Immediate có hợp lệ không.
        \item Gọi lại hàm \texttt{Split\_Sign\_ExtImm} để tiếp tục xử lý phần còn lại của chuỗi lệnh.
        \item Gọi hàm \texttt{Check\_Parentheses\_1} để kiểm tra ký tự \texttt{'('}.
        \item Gọi lại hàm \texttt{Split\_Sign\_ExtImm} để tiếp tục xử lý phần còn lại của chuỗi lệnh.
        \item Gọi hàm \texttt{Check\_Register} để kiểm tra xem thanh ghi có hợp lệ không.
        \item Gọi lại hàm \texttt{Split\_Sign\_ExtImm} để tiếp tục xử lý phần còn lại của chuỗi lệnh.
        \item Gọi hàm \texttt{Check\_Parentheses\_2} để kiểm tra ký tự \texttt{')'}.
    \end{itemize}

    \item \textbf{Kiểm tra ký tự cuối cùng:}
    \begin{itemize}
        \item Kiểm tra ký tự cuối cùng được lưu trong \texttt{\$t9}.
        \item Nếu ký tự cuối cùng là newline (\texttt{10}), nhảy đến \texttt{Right\_code}.
        \item Nếu ký tự cuối cùng là NULL (\texttt{0}), nhảy đến \texttt{Right\_code}.
        \item Nếu ký tự cuối cùng là dấu \texttt{')'} (\texttt{41}), nhảy đến \texttt{Right\_code}.
        \item Nếu không, nhảy đến \texttt{False\_code}.
    \end{itemize}
\end{itemize}
\subsection{In ra kết quả}

Các xâu kí tự từ \texttt{Message 1 - Message 11} đã được khai báo và giải thích ở mục \textbf{2.3.1 Khai báo dữ liệu.}

\subsubsection{a. In opcode}
\textbf{Trường hợp opcode đúng thoả mãn yêu cầu đề bài}
\begin{lstlisting}
    False_opcode:
	#Print "Opcode"
	li $v0, 4
	la $a0, Message2
	syscall
	nop
	#Print Opcode Input
	li $v0, 4
	add $a0, $zero, $s2
	syscall
	nop
	#Print "Khong hop le!"
	li $v0, 4
	la $a0, Message4
	syscall
	nop
	jal	False_code
	j	End_main
\end{lstlisting}

\textbf{Trường hợp opcode sai}
\begin{lstlisting}
    Right_opcode:
	#Print "Opcode"
	li $v0, 4
	la $a0, Message2
	syscall
	nop
	#Print Opcode Input
	li $v0, 4
	add $a0, $zero, $s2
	syscall
	nop
	#Print ", hop le!"
	li $v0, 4
	la $a0, Message3
	syscall
	nop
	jr	$ra
\end{lstlisting}

\subsubsection{b. In thanh ghi, số, nhãn label}
Ta xét các trường hợp thanh ghi, số, nhãn thoả mãn đề bài.
\begin{lstlisting}
    Right_Register:
	#Print "\n"
	li $v0, 4
	la $a0, Message7
	syscall
	nop
	#Print "Thanh ghi"
	li $v0, 4
	la $a0, Message8
	syscall
	nop
	#Print Register Input
	li $v0, 4
	add $a0, $zero, $s2
	syscall
	nop
	#Print ", hop le!"
	li $v0, 4
	la $a0, Message3
	syscall
	nop
	jr	$ra
Print_Right_Number:
	#Print "\n"
	li $v0, 4
	la $a0, Message7
	syscall
	nop
	#Print "So "
	li $v0, 4
	la $a0, Message9
	syscall
	nop
	#Print so trong hang doi
	li $v0, 4
	add $a0, $zero, $s2
	syscall
	nop
	#Print ", hop le!"
	li $v0, 4
	la $a0, Message3
	syscall
	nop
	jr	$ra
Print_Right_Label:
	#Print "\n"
	li $v0, 4
	la $a0, Message7
	syscall
	nop
	#Print "So "
	li $v0, 4
	la $a0, Message10
	syscall
	nop
	#Print label trong hang doi
	li $v0, 4
	add $a0, $zero, $s2
	syscall
	nop
	#Print ", hop le!"
	li $v0, 4
	la $a0, Message3
	syscall
	nop
	jr	$ra
\end{lstlisting}

\subsubsection{c. In khẳng định câu lệnh đúng/ sai}
\begin{lstlisting}
    Right_Register:
	#Print "\n"
	li $v0, 4
	la $a0, Message7
	syscall
	nop
	#Print "Thanh ghi"
	li $v0, 4
	la $a0, Message8
	syscall
	nop
	#Print Register Input
	li $v0, 4
	add $a0, $zero, $s2
	syscall
	nop
	#Print ", hop le!"
	li $v0, 4
	la $a0, Message3
	syscall
	nop
	jr	$ra
Print_Right_Number:
	#Print "\n"
	li $v0, 4
	la $a0, Message7
	syscall
	nop
	#Print "So "
	li $v0, 4
	la $a0, Message9
	syscall
	nop
	#Print so trong hang doi
	li $v0, 4
	add $a0, $zero, $s2
	syscall
	nop
	#Print ", hop le!"
	li $v0, 4
	la $a0, Message3
	syscall
	nop
	jr	$ra
Print_Right_Label:
	#Print "\n"
	li $v0, 4
	la $a0, Message7
	syscall
	nop
	#Print "So "
	li $v0, 4
	la $a0, Message10
	syscall
	nop
	#Print label trong hang doi
	li $v0, 4
	add $a0, $zero, $s2
	syscall
	nop
	#Print ", hop le!"
	li $v0, 4
	la $a0, Message3
	syscall
	nop
	jr	$ra
\end{lstlisting}

\subsubsection{d. Đưa ra câu hỏi và lặp lại chương trình}
\begin{lstlisting}
    Run_Again:	li $v0, 50
		la $a0, Message11
		syscall
		nop
		beq $a0, $zero, clear
		nop
		j exit
		nop
# clear: dua string ve trang thai ban dau de thuc hien lai qua trinh
clear:		add	$s3, $zero, $s1	
Loop_Null: 
		lb	$t3, 0($s3)
		li 	$t5, 10	
		beq	$t3, $t5, Loop_Null_them
		nop
		sb	$zero, 0($s3)
		addi 	$s3, $s3, 1
		j	Loop_Null
Loop_Null_them:
		sb	$zero, 0($s3)
		j start
		nop
	
\end{lstlisting}
\textbf{Giải thích}
\begin{itemize}
    \item \textbf{Run\_Again:}
    \begin{itemize}
        \item Hiển thị thông báo hỏi người dùng có muốn tiếp tục kiểm tra lệnh khác hay không.
        \item Nếu người dùng nhập 0 (nghĩa là không muốn tiếp tục), chương trình nhảy đến nhãn \texttt{clear}.
        \item Nếu không, chương trình kết thúc.
    \end{itemize}
    \item \textbf{clear:}
    \begin{itemize}
        \item Xóa nội dung của chuỗi lệnh đã kiểm tra bằng cách thay thế từng ký tự trong chuỗi bằng \texttt{NULL}.
    \end{itemize}
    \item \textbf{Loop\_Null:}
    \begin{itemize}
        \item Vòng lặp xóa các ký tự trong chuỗi lệnh. Nếu gặp ký tự newline, nhảy đến nhãn \texttt{Loop\_Null\_them} để xóa ký tự newline và sau đó quay lại đầu chương trình (\texttt{start}) để người dùng nhập lệnh mới.
    \end{itemize}
\end{itemize}

Đoạn mã này thực hiện chức năng hỏi người dùng có muốn tiếp tục kiểm tra lệnh khác hay không. Nếu người dùng chọn tiếp tục, chương trình sẽ xóa nội dung của chuỗi lệnh đã kiểm tra và quay lại từ đầu để người dùng nhập lệnh mới. Nếu người dùng chọn không tiếp tục, chương trình sẽ kết thúc.

\section{Kết quả}


\subsection*{Dạng lệnh R (Register)}
\subsubsection*{Hợp lệ}
\begin{itemize}
    \item \texttt{add \$t1, \$t2, \$t3} \\ Thêm giá trị của thanh ghi \texttt{\$t2} và \texttt{\$t3}, lưu kết quả vào \texttt{\$t1}.
    
\end{itemize}

\subsubsection*{Không hợp lệ}
\begin{itemize}
    \item \texttt{add \$t1, \$t2, \$s32} \\ Lỗi vì \texttt{\$s32} không phải là thanh ghi hợp lệ.
    \item \texttt{mul \$t1 \$t2 \$t3} \\ Lỗi vì thiếu dấu phẩy giữa các tham số.
\end{itemize}
\begin{figure}[H]
	\begin{framed}
		\centering
		\includegraphics[width=15.0cm]{images/kq_R.png}
		\caption{Kết quả chạy 1 số câu lệnh loại R}
		\label{s2attn_R}
	\end{framed}
\end{figure}
\subsection*{Dạng lệnh I (Immediate)}
\subsubsection*{Hợp lệ}
\begin{itemize}
    \item \texttt{addi \$t1, \$t2, 10} \\ Thêm giá trị tức thời \texttt{10} vào giá trị của thanh ghi \texttt{\$t2}, lưu kết quả vào \texttt{\$t1}.
\end{itemize}

\subsubsection*{Không hợp lệ}
\begin{itemize}
    \item \texttt{addi \$t1, \$t2, 0xG} \\ Lỗi vì \texttt{0xG} không phải là giá trị tức thời hợp lệ.
    \item \texttt{addi \$t1, \$t2 10} \\ Lỗi vì thiếu dấu phẩy giữa các tham số.
\end{itemize}
\begin{figure}[H]
	\begin{framed}
		\centering
		\includegraphics[width=15.0cm]{images/kq_I.png}
		\caption{Kết quả chạy 1 số câu lệnh loại I}
		\label{s2attn_I}
	\end{framed}
\end{figure}
\subsection*{Dạng lệnh J (Jump)}
\subsubsection*{Hợp lệ}
\begin{itemize}
    \item \texttt{jal 0x00400000} \\ Nhảy tới địa chỉ \texttt{0x00400000} và lưu địa chỉ của lệnh tiếp theo vào thanh ghi \texttt{\$ra}.
\end{itemize}

\subsubsection*{Không hợp lệ}
\begin{itemize}
    \item \texttt{jal00400000} \\ Lỗi vì thiếu khoảng cách giữa opcode và địa chỉ.
\end{itemize}
\begin{figure}[H]
	\begin{framed}
		\centering
		\includegraphics[width=15.0cm]{images/kq_j.png}
		\caption{Kết quả chạy 1 số câu lệnh loại J}
		\label{s2attn_J}
	\end{framed}
\end{figure}
\subsection*{Dạng lệnh L (Load/Store)}
\subsubsection*{Hợp lệ}
\begin{itemize}
    \item \texttt{lw \$t1, 4(\$t2)} \\ Tải giá trị từ địa chỉ tính bằng giá trị của thanh ghi \texttt{\$t2} cộng với \texttt{4} vào thanh ghi \texttt{\$t1}.

\end{itemize}

\subsubsection*{Không hợp lệ}
\begin{itemize}
    \item \texttt{lw \$t1 \$t2 4} \\ Lỗi vì thiếu dấu phẩy và định dạng không đúng.
    \item \texttt{sw \$s1, (\$s2)8} \\ Lỗi vì thứ tự các tham số không đúng.
\end{itemize}
\begin{figure}[H]
	\begin{framed}
		\centering
		\includegraphics[width=15.0cm]{images/kq_l.png}
		\caption{Kết quả chạy 1 số câu lệnh loại L}
		\label{s2attn_L}
	\end{framed}
\end{figure}
\subsection*{Dạng lệnh đặc biệt (Special)}
\subsubsection*{Hợp lệ}
\begin{itemize}
    \item \texttt{syscall} \\ Gọi hệ thống.
    \item \texttt{nop} \\ Không làm gì (\textit{no operation}).
\end{itemize}
\begin{figure}[H]
	\begin{framed}
		\centering
		\includegraphics[width=15.0cm]{images/kq_db.png}
		\caption{Kết quả chạy 1 số câu lệnh đặc biệt}
		\label{s2attn_special}
	\end{framed}
\end{figure}

\section{Source Code Full}
