\chapter{REQUIREMENT ANALYSIS}
\section{Problem Definition}

\subsection{Background and Motivation}
Students today face difficulties in finding suitable learning tools. Existing assessment and learning platforms on the market such as \textbf{Quizizz, Azota, and Google Forms} still have several limitations in terms of:
\begin{itemize}
    \item \textbf{Personalization:} Lack of mechanisms to tailor learning content to each individual.
    \item \textbf{Feedback:} Feedback is often simple and does not provide in-depth analysis.
    \item \textbf{Interaction:} The learning environment is not flexible or engaging enough.
\end{itemize}
\textbf{QuizSpark} was created with the goal of overcoming these weaknesses, providing a more flexible and intelligent learning environment that meets the growing demand for personalized education.

\subsection{Problem Description}
The core problem that QuizSpark aims to address can be specified as follows:
\begin{itemize}
    \item \textbf{Lack of Adaptive Difficulty:} Existing platforms only provide static quizzes. They lack the ability to automatically adjust the difficulty level of questions based on the learner’s actual competency and learning progress. 
    \item \textbf{Unverified Content Quality:} The question pool mainly comes from user-generated content (UGC), leading to a lack of consistency and no strict quality validation in terms of academic standards.
    \item \textbf{Shallow Feedback:} Post-assessment feedback usually stops at Right/Wrong or total score. It does not provide in-depth analysis of weak knowledge areas, common mistakes, or suggestions for the next learning steps.
\end{itemize}

\subsection{Project Objectives}
The QuizSpark project is guided by the following strategic objectives:
\begin{itemize}
    \item \textbf{Create a Learner-Centric Experience:} Design an intuitive and user-friendly interface and learning flow that prioritize students’ needs and experience.
    \item \textbf{Integrate an Adaptive Learning Mechanism:} Implement algorithms to personalize the sequence of questions and learning materials, ensuring learners always face challenges suitable to their current level ($\pm$ Zone of Proximal Development - ZPD).
    \item \textbf{Build an Analytics Dashboard:} Develop a visual interface that allows learners and teachers to track detailed results, perform in-depth analysis of strengths and weaknesses, and generate personalized learning recommendations.
\end{itemize}

\section{Stakeholder Analysis}
Analyzing stakeholders is essential to ensure that \textbf{QuizSpark} effectively meets the needs of all user groups and system operators.

\begin{table}[h]
\centering
\label{tab:stakeholder}
\begin{tabular}{|p{3cm}|p{5cm}|p{5cm}|}
\hline
\textbf{Stakeholder} & \textbf{Role / Objective} & \textbf{Main Needs} \\
\hline
Students / Learners & Primary users of the system. Their goal is to improve academic performance through personalized learning paths. & Take adaptive quizzes, track detailed learning progress, and receive study suggestions based on weaknesses.\\
\hline
Teachers / Tutors & Content creators and managers. Their goal is to enhance teaching and assessment effectiveness. & Manage classes, create high-quality exams, and view student statistics and class performance analytics.\\
\hline
Developers & Build and deploy system features. Their goal is performance and reliability. & Design scalable, maintainable code and integrate AI/ML technologies.\\
\hline
System Administrators (Admins) & Manage users and content across the platform. Their goal is safety and compliance. & Moderate content, secure user data, and monitor system performance and stability.\\
\hline
Advertising Partners & Provide revenue for the platform. Their goal is to reach appropriate target audiences. & Implement contextual advertising mechanisms suitable for students, ensuring ads are non-intrusive and do not disrupt the learning experience.\\
\hline
\end{tabular}
\end{table}
So, the need of \textbf{Students} (Adaptive Testing, Detailed Progress Tracking) are central, directly influencing Objective 1.3. The needs of \textbf{Teachers} (Content Creation, Analytics) determine the usefulness of the tool in real educational environments.

\section{System Overview}
\subsection{Data Flow Diagram}

\begin{figure}[H]
    \centering
    \includegraphics[width=\textwidth]{images/DFD.png}
    \caption{Data Flow Diagram for QuizSpark System}
    \label{fig:dfd_quizspark}
\end{figure}

\subsection{State Diagram}

\begin{figure}[H]
    \centering
    \includegraphics[width=\textwidth]{images/StateDiagram.png}
    \caption{State Diagram for QuizSpark System}
    \label{fig:state_quizspark}
\end{figure}

\subsection{Entity Relationship Diagram}
\begin{figure}[H]
    \centering
    \includegraphics[width=\textwidth]{images/EntityDiagram.png}
    \caption{Entity Relationship Diagram for QuizSpark System}
    \label{fig:erd_quizspark}
\end{figure}


\section{Requirements}
\subsection{Students \& Teacher Functional Requirements}

\paragraph{FR-U01 - Register/Login}
Allows both students and teachers to register a new account or log in to the system using existing credentials.

\begin{figure}[H]
    \centering
    \includegraphics[width=0.8\textwidth]{images/FR-U01.png}
    \caption{Sequence Diagram - FR-U01 - Register/Login}
\end{figure}

\noindent\begin{table}[H]
\centering
\label{tab:uc_frs01_final}
\begin{tabular}{|p{3cm}|p{13cm}|}
\hline
\textbf{Component} & \textbf{Content} \\
\hline
Requirements & Register / Login \\
\hline
Code & FR-U01 \\
\hline
Main Actor & Student \& Teacher \\
\hline
Goal & Enable users (students and teachers) to create a new account or log in to an existing one on QuizSpark. \\
\hline
Precondition & The user has not logged in (no active session). \\
\hline
Postcondition & The user is successfully authenticated and redirected to their Dashboard. \\
\hline
Main Flow & \vspace{-0.3 cm}
\begin{enumerate}[nosep, leftmargin=*]
    \item The user selects ``Register'' or ``Login'' on the interface.
    \item The user enters account information (e.g., email/username and password).
    \item The system validates the input data (format, existence, and password match).
    \item If valid, the system creates a login session.
    \item The system displays the user's Dashboard.
\end{enumerate} \\
\hline
Alternative Flow & \vspace{-0.3 cm}
\begin{itemize}[nosep, leftmargin=*]
    \item \textbf{3a: Invalid information:} If the input is invalid or does not exist, the system displays an error message and asks for re-entry.
    \item \textbf{4a: Unverified account:} If the account exists but has not been verified via email, the system requests email verification before allowing login.
\end{itemize} \\
\hline
\end{tabular}
\end{table}

\paragraph{FR-U02 - Create, Edit, Delete Personal Quiz}
Allows both students and teachers to create, modify, or delete their own quizzes for practice or sharing purposes.

\begin{figure}[H]
    \centering
    \includegraphics[width=0.8\textwidth]{images/FR-U02.png}
    \caption{Activity Diagram - FR-U02 - Manage Personal Quiz}
\end{figure}

\noindent\begin{table}[H]
\centering
\label{tab:uc_fru02_final}
\begin{tabular}{|p{3cm}|p{13cm}|}
\hline
\textbf{Component} & \textbf{Content} \\
\hline
Requirements & Create, Edit, Delete Personal Quiz \\
\hline
Code & FR-U02 \\
\hline
Main Actor & Student \& Teacher \\
\hline
Goal & Allow users (students and teachers) to manage their own quizzes for practice, teaching, or sharing purposes. \\
\hline
Precondition & The user has logged in (FR-U01). \\
\hline
Postcondition & The created or updated quiz is stored in the database; deleted quizzes are permanently removed. \\
\hline
Main Flow & \vspace{-0.3cm}
\begin{enumerate}[nosep, leftmargin=*]
    \item The user navigates to ``My Quizzes'' and chooses to create, edit, or delete a quiz.
    \item For creation: the system displays a form to input quiz title, description, visibility (private/public), and questions.
    \item The user enters quiz information and adds questions (multiple choice, short answer, etc.).
    \item The user saves the quiz; the system validates and stores it in the database.
    \item For editing: the user modifies existing quiz details or questions and saves the changes.
    \item For deletion: the system asks for confirmation before permanently deleting the selected quiz.
\end{enumerate} \\
\hline
Alternative Flow & \vspace{-0.3cm}
\begin{itemize}[nosep, leftmargin=*]
    \item \textbf{2a: Missing information:} If required fields are left empty, the system notifies the user to complete them.
    \item \textbf{4a: Invalid input:} If question formats or inputs are invalid, the system rejects the quiz and highlights the errors.
    \item \textbf{6a: Deletion canceled:} If the user cancels the deletion, the quiz remains unchanged.
\end{itemize} \\
\hline
\end{tabular}
\end{table}



\paragraph{FR-U03 - View Leaderboard}
Display the leaderboard showing users’ scores and achievements within a class or across the entire system to encourage positive competition and engagement.

\noindent
\begin{table}[H]
\centering
\label{tab:uc_frs04_final}
\begin{tabular}{|p{3cm}|p{13cm}|}
\hline
\textbf{Component} & \textbf{Content} \\
\hline
Requirements & View Leaderboard \\
\hline
Code & FR-U03 \\
\hline
Main Actor & Student \& Teacher \\
\hline
Goal & Allow users (students and teachers) to view leaderboards that display scores and rankings to enhance motivation and competitive learning. \\
\hline
Precondition & The system already has quiz result data from multiple users. \\
\hline
Postcondition & Display a leaderboard sorted by score, completion time, or quiz performance. \\
\hline
Main Flow & \vspace{-0.3cm}
\begin{enumerate}[nosep, leftmargin=*]
    \item The user selects the “Leaderboard” option from the interface.
    \item The system retrieves quiz results and achievements from the database.
    \item The system sorts the data by ranking.
    \item The leaderboard is displayed with user name, score, and rank (Top 1, Top 10, etc.).
    \item The user can filter the leaderboard by class or system-wide.
\end{enumerate} \\
\hline
Alternative Flow & \vspace{-0.3cm}
\begin{itemize}[nosep, leftmargin=*]
    \item \textbf{2a: No data:} If no quiz results are available, the system displays the message “No leaderboard data available.”
\end{itemize} \\
\hline
\end{tabular}
\end{table}

\paragraph{FR-U04 - Generate Questions (AI)}
Allow users to utilize the AI feature to automatically generate questions based on topic, difficulty level, and desired format, supporting faster quiz creation and learning preparation.

\noindent
\begin{table}[H]
\centering
\label{tab:uc_frs05_final}
\begin{tabular}{|p{3cm}|p{13cm}|}
\hline
\textbf{Component} & \textbf{Content} \\
\hline
Requirements & Generate Questions (AI) \\
\hline
Code & FR-U04 \\
\hline
Main Actor & Student \& Teacher \\
\hline
Goal & Enable users (students and teachers) to use AI to automatically generate questions by topic, difficulty level, and preferred format to assist in quiz creation or study practice. \\
\hline
Precondition & The user is logged in and has access to the AI question generation tool. The AI system is functioning properly. \\
\hline
Postcondition & The system displays a list of AI-generated questions for the user to preview, modify, and select for inclusion in a quiz or study set. \\
\hline
Main Flow & \vspace{-0.3cm}
\begin{enumerate}[nosep, leftmargin=*]
    \item The user opens the “AI Question Generator” tool.
    \item The user enters the topic, difficulty level, question type, and desired number of questions.
    \item The AI system processes the request and generates a list of questions.
    \item The system displays the generated questions for the user to review.
    \item The user selects or edits suitable questions to add to the quiz or study set.
    \item The system saves the selected questions into the corresponding quiz or collection.
\end{enumerate} \\
\hline
Alternative Flow & \vspace{-0.3cm}
\begin{itemize}[nosep, leftmargin=*]
    \item \textbf{3a: Missing input data:} The system displays the message “Please fill in all required fields before generating questions.”
    \item \textbf{3b: AI error or generation failure:} The system displays the message “Unable to generate questions. Please try again later.”
\end{itemize} \\
\hline
\end{tabular}
\end{table}

\paragraph{FR-U05 - Multi-Platform Access}
Allow users to access and use the QuizSpark platform seamlessly on both web and mobile devices with synchronized data and consistent user experience.

\noindent
\begin{table}[H]
\centering
\label{tab:uc_frs06_final}
\begin{tabular}{|p{3cm}|p{13cm}|}
\hline
\textbf{Component} & \textbf{Content} \\
\hline
Requirements & Access the platform on both web and mobile devices \\
\hline
Code & FR-U05 \\
\hline
Main Actor & Student \& Teacher \\
\hline
Goal & Enable users (students and teachers) to access the QuizSpark platform from web browsers or mobile applications while maintaining data consistency and usability. \\
\hline
Precondition & The user has a registered account and a stable internet connection. \\
\hline
Postcondition & The user can log in and use the system across devices with synchronized progress and quiz data. \\
\hline
Main Flow & \vspace{-0.3cm}
\begin{enumerate}[nosep, leftmargin=*]
    \item The user opens QuizSpark on a web browser or mobile app.
    \item The user logs in using their account credentials (FR-U01).
    \item The system retrieves user data (quizzes, results, progress) from the database.
    \item The system adapts the interface layout and functionality according to the device type.
    \item The user can access all core features, including taking quizzes, viewing results, and managing personal content.
    \item Any actions performed on one device (e.g., creating a quiz, submitting results) are synchronized across all devices.
\end{enumerate} \\
\hline
Alternative Flow & \vspace{-0.3cm}
\begin{itemize}[nosep, leftmargin=*]
    \item \textbf{1a: Unsupported device or browser:} The system displays a message suggesting the use of supported platforms or the latest version.
    \item \textbf{2a: Connection error:} If the internet connection fails, the system switches to limited offline mode (if available) and syncs data when reconnected.
    \item \textbf{4a: Display issue:} If the interface does not load properly on the device, the user is prompted to refresh or update the application.
\end{itemize} \\
\hline
\end{tabular}
\end{table}

\paragraph{FR-U06 - Share Quizzes}
Allow users to share created quizzes publicly or with selected users (students or teachers) through unique links or controlled access permissions.

\noindent
\begin{table}[H]
\centering
\label{tab:uc_frs07_final}
\begin{tabular}{|p{3cm}|p{13cm}|}
\hline
\textbf{Component} & \textbf{Content} \\
\hline
Requirements & Share quizzes publicly or with selected users \\
\hline
Code & FR-U06 \\
\hline
Main Actor & Student \& Teacher \\
\hline
Goal & Enable users to share their created quizzes with others — either publicly for everyone or privately with specific users. \\
\hline
Precondition & The user is logged in and has at least one quiz created. \\
\hline
Postcondition & The quiz becomes available to the selected audience according to the chosen sharing option. \\
\hline
Main Flow & \vspace{-0.3cm}
\begin{enumerate}[nosep, leftmargin=*]
    \item The user opens the “My Quizzes” section.
    \item The user selects a quiz to share.
    \item The system displays sharing options: “Public” or “Selected Users”.
    \item The user chooses a sharing method:
    \begin{itemize}[nosep, leftmargin=*]
        \item \textbf{Public:} The quiz is accessible to all users through a public link or the shared library.
        \item \textbf{Selected Users:} The user specifies recipient accounts (students/teachers) or enters email addresses.
    \end{itemize}
    \item The system generates a unique access link or updates access permissions.
    \item The recipients receive a notification or can access the quiz via the provided link.
\end{enumerate} \\
\hline
Alternative Flow & \vspace{-0.3cm}
\begin{itemize}[nosep, leftmargin=*]
    \item \textbf{2a: No quiz available:} The system notifies the user that there are no quizzes to share.
    \item \textbf{4a: Invalid recipient:} If entered user information is invalid, the system prompts re-entry.
    \item \textbf{5a: Permission error:} If the user lacks sharing privileges, the system restricts access and displays an error message.
\end{itemize} \\
\hline
\end{tabular}
\end{table}

\subsection{Student Functional Requirements}
\paragraph{FR-S01 - Join Quiz Session}
Allow students to join real-time competitive quiz sessions hosted by teachers using a session code or link.


\begin{figure}[H]
    \centering
    \includegraphics[width=0.9\textwidth]{images/FR-S01.png}
    \caption{Sequence Diagram - FR-S01 - Join Quiz Session}
\end{figure}

\noindent
\begin{table}[H]
\centering
\label{tab:uc_frs08_final}
\begin{tabular}{|p{3cm}|p{13cm}|}
\hline
\textbf{Component} & \textbf{Content} \\
\hline
Requirements & Join a competitive quiz session \\
\hline
Code & FR-S01 \\
\hline
Main Actor & Student \\
\hline
Goal & Allow students to participate in real-time quiz sessions using a session code or invitation link provided by the teacher. \\
\hline
Precondition & The student is logged in and has a valid session code or access link. \\
\hline
Postcondition & The student successfully joins the live quiz session and can answer questions in real time. \\
\hline
Main Flow & \vspace{-0.3cm}
\begin{enumerate}[nosep, leftmargin=*]
    \item The student selects “Join Quiz” from the dashboard.
    \item The student enters the session code or clicks the invitation link.
    \item The system verifies the session code and participant eligibility.
    \item The student joins the quiz lobby and waits for the host to start the session.
    \item Once started, the system displays questions one by one for the student to answer.
    \item The student submits answers, and the system updates their real-time score.
\end{enumerate} \\
\hline
Alternative Flow & \vspace{-0.3cm}
\begin{itemize}[nosep, leftmargin=*]
    \item \textbf{2a: Invalid or expired code:} The system displays “Invalid session code. Please check and try again.”
    \item \textbf{3a: Session full or closed:} The system displays “Unable to join. The session is full or has ended.”
\end{itemize} \\
\hline
\end{tabular}
\end{table}

\paragraph{FR-S02 - Submit Quiz \& View Results}
After completing a quiz, users can submit their answers. The system automatically grades the submission, stores results, and provides detailed analytics for performance review.

\noindent
\begin{table}[H]
\centering
\label{tab:uc_frs02_final}
\begin{tabular}{|p{3cm}|p{13cm}|}
\hline
\textbf{Component} & \textbf{Content} \\
\hline
Requirements & Submit Quiz \& View Results \\
\hline
Code & FR-S02 \\
\hline
Main Actor & Student \\
\hline
Goal & Allow students to submit completed quizzes, automatically receive grading results, and review performance analytics. \\
\hline
Precondition & The user has completed all questions in a valid quiz (FR-U02). \\
\hline
Postcondition & The system grades the quiz, stores detailed results (score, time, answers) in the database, and displays analytics for the user. \\
\hline
Main Flow & \vspace{-0.3cm}
\begin{enumerate}[nosep, leftmargin=*]
    \item The user selects ``Submit Quiz'' after answering all questions.
    \item The system asks for submission confirmation.
    \item Upon confirmation, the system collects answers, locks editing, and sends data to the server.
    \item The system automatically grades the submission based on the question bank.
    \item The system records results (score, completion time, and detailed responses) in the database.
    \item The system displays the result screen, including:
    \begin{itemize}[nosep, leftmargin=*]
        \item Total score and completion time.
        \item Correct/incorrect answers and explanations (if available).
        \item Performance charts summarizing strengths and weaknesses by topic.
    \end{itemize}
    \item The user can download or review detailed reports for further assessment or feedback.
\end{enumerate} \\
\hline
Alternative Flow & \vspace{-0.3cm}
\begin{itemize}[nosep, leftmargin=*]
    \item \textbf{2a: Submission canceled:} If the user cancels, the system returns to the quiz interface.
    \item \textbf{3a: Connection lost:} If the network disconnects during submission, progress is saved temporarily and resubmitted automatically once restored.
    \item \textbf{4a: Grading error:} If automatic grading fails, the quiz is marked as “Pending Review” for manual evaluation by the teacher.
    \item \textbf{6a: Missing data:} If some analytics data cannot be retrieved, the system shows available results with a warning message.
\end{itemize} \\
\hline
\end{tabular}
\end{table}

\subsection{Teacher Functional Requirements}

\paragraph{FR-T01 - Manage Classes}
Teachers can manage class lists, including adding or removing students and assigning quizzes to classes.

\noindent
\begin{tabular}{|p{3cm}|p{13cm}|}
\hline
\textbf{Component} & \textbf{Content} \\
\hline
Requirements & Manage Classes \\
\hline
Code & FR-T01 \\
\hline
Main Actor & Teacher \\
\hline
Goal & Manage class lists by adding or removing students and assigning quizzes to each class to monitor learning progress. \\
\hline
Precondition & The teacher is logged in and has at least one class created in the system. \\
\hline
Postcondition & The class list is updated, students are added or removed, and quizzes are assigned to the corresponding class. \\
\hline
Main Flow & \vspace{-0.3cm}
\begin{enumerate}[nosep, leftmargin=*]
    \item The teacher accesses the “Class Management” section.
    \item The system displays the list of existing classes.
    \item The teacher selects a specific class to edit.
    \item The teacher can perform the following actions: add a new student, remove a student, or assign a quiz to the class.
    \item The system saves the changes and displays a confirmation message.
\end{enumerate} \\
\hline
Alternative Flow & \vspace{-0.3cm}
\begin{itemize}[nosep, leftmargin=*]
    \item \textbf{4a: Student not found:} When adding a student with an invalid ID or email, the system displays “Student not found.”
    \item \textbf{4b: Quiz assignment error:} If the system cannot assign the quiz due to network or data issues, it displays “Unable to assign quiz. Please try again later.”
\end{itemize} \\
\hline
\end{tabular}

\paragraph{FR-T02 - Host Quiz Session}
Allow teachers to host and manage real-time competitive quiz sessions, inviting students to join via a session code or link.

\begin{figure}[H]
    \centering
    \includegraphics[width=0.9\textwidth]{images/FR-T02.png}
    \caption{Activity Diagram - FR-T02 - Host Quiz Session}
\end{figure}

\noindent
\begin{table}[H]
\centering
\label{tab:uc_frt02_final}
\begin{tabular}{|p{3cm}|p{13cm}|}
\hline
\textbf{Component} & \textbf{Content} \\
\hline
Requirements & Host a competitive quiz session \\
\hline
Code & FR-T02 \\
\hline
Main Actor & Teacher \\
\hline
Goal & Enable teachers to host interactive quiz sessions, monitor participation, and view results in real time. \\
\hline
Precondition & The teacher is logged in and has at least one quiz ready to host. \\
\hline
Postcondition & The system starts a live session and allows students to join using a unique code or link. \\
\hline
Main Flow & \vspace{-0.3cm}
\begin{enumerate}[nosep, leftmargin=*]
    \item The teacher opens the “Host Quiz” feature.
    \item The teacher selects a quiz to host.
    \item The system generates a unique session code and invitation link.
    \item The teacher shares the code/link with students.
    \item Students join the session, and the teacher monitors participant status.
    \item The teacher starts the quiz, and the system presents questions to all participants.
    \item The teacher views live responses and leaderboard updates.
    \item After completion, the system displays summary results for the teacher and students.
\end{enumerate} \\
\hline
Alternative Flow & \vspace{-0.3cm}
\begin{itemize}[nosep, leftmargin=*]
    \item \textbf{2a: No quiz available:} The system displays “Please create or select a quiz before hosting.”
    \item \textbf{3a: Network issue or timeout:} The system shows “Unable to start session. Please check your connection.”
\end{itemize} \\
\hline
\end{tabular}
\end{table}



\subsection{Admin Functional Requirements}

\paragraph{FR-A01 - Manage Users}
The administrator has the authority to add, block/deactivate, or delete user accounts (students and teachers) from the system.

\noindent
\begin{tabular}{|p{3cm}|p{13cm}|}
\hline
\textbf{Component} & \textbf{Description} \\
\hline
Requirements & Manage Users \\
\hline
Code & FR-A01 \\
\hline
Primary Actor & \textbf{Admin} \\
\hline
Goal & The administrator manages user accounts, including adding new users, blocking/deactivating, or deleting student and teacher accounts to maintain stable and secure system operation. \\
\hline
Preconditions & The administrator is logged in and has access to the admin dashboard. \\
\hline
Postconditions & The system updates the user’s status or information accordingly (added, blocked, or deleted). \\
\hline
Main Flow & \vspace{-0.3cm}
\begin{enumerate}[nosep, leftmargin=*]
    \item The administrator navigates to the “User Management” section.
    \item The system displays the list of current users.
    \item The administrator selects an action: add new, block, or delete an account.
    \item The administrator enters or confirms the required information.
    \item The system applies the changes and displays a confirmation message.
\end{enumerate} \\
\hline
Alternative Flow & \vspace{-0.3cm}
\begin{itemize}[nosep, leftmargin=*]
    \item \textbf{4a: Invalid information entered:} The system displays the message “Invalid user information.”
    \item \textbf{5a: Connection or data saving error:} The system displays the message “Unable to update information, please try again.”
\end{itemize} \\
\hline
\end{tabular}


\paragraph{FR-A02 - Moderate Content}
The administrator performs moderation and management of quiz content created by users or inappropriate feedback/comments on the platform.

\noindent
\begin{tabular}{|p{3cm}|p{13cm}|}
\hline
\textbf{Component} & \textbf{Description} \\
\hline
Requirements & Moderate Content \\
\hline
Code & FR-A02 \\
\hline
Primary Actor & \textbf{Admin} \\
\hline
Goal & The administrator moderates quizzes, feedback, or inappropriate comments to ensure the quality and safety of the platform. \\
\hline
Preconditions & The administrator is logged in and has access to the content moderation tool. \\
\hline
Postconditions & Violating content is deleted, hidden, or flagged for review; valid content remains unchanged. \\
\hline
Main Flow & \vspace{-0.3cm}
\begin{enumerate}[nosep, leftmargin=*]
    \item The administrator navigates to the “Content Moderation” section.
    \item The system displays a list of recent quizzes, comments, or feedback.
    \item The administrator reviews reported or potentially violating content.
    \item The administrator selects an action: keep, hide, or delete the content.
    \item The system updates the status and logs the moderation history.
\end{enumerate} \\
\hline
Alternative Flow & \vspace{-0.3cm}
\begin{itemize}[nosep, leftmargin=*]
    \item \textbf{3a: No content to moderate:} The system displays the message “No violating content found.”
    \item \textbf{5a: Processing error:} If the system cannot delete or update the status, it displays the message “Unable to perform the action, please try again.”
\end{itemize} \\
\hline
\end{tabular}


\subsection{AI Engine Functional Requirements}

\paragraph{FR-AI01 - Auto-generate Questions}
The AI Engine must be able to automatically generate new and valid questions based on input topics and difficulty levels, supporting the enrichment of the question database.

\begin{figure}[H]
    \centering
    \includegraphics[width=0.95\textwidth]{images/FR-AI01.png}
    \caption{Sequence Diagram - FR-AI01 - Generate Questions (AI)}
\end{figure}

\noindent
\begin{tabular}{|p{3cm}|p{13cm}|}
\hline
\textbf{Component} & \textbf{Description} \\
\hline
Requirements & Auto-generate Questions \\
\hline
Code & FR-AI01 \\
\hline
Primary Actor & \textbf{AI Engine} \\
\hline
Goal & Automatically generate valid questions based on the provided topic and difficulty level, in order to expand the question database for learning and assessment purposes. \\
\hline
Preconditions & The AI system is running properly and has access to the linguistic data and input topics. \\
\hline
Postconditions & A list of newly generated questions is stored in the database and ready for teachers to use. \\
\hline
Main Flow & \vspace{-0.3cm}
\begin{enumerate}[nosep, leftmargin=*]
    \item The AI receives a request to generate questions along with input parameters (topic, difficulty, format).
    \item The AI processes the request, analyzes the content, and generates questions.
    \item The system validates the generated questions and checks for duplication.
    \item The system saves valid questions to the database.
\end{enumerate} \\
\hline
Alternative Flow & \vspace{-0.3cm}
\begin{itemize}[nosep, leftmargin=*]
    \item \textbf{2a: Insufficient input data:} The system displays the message “Insufficient data to generate questions.”
    \item \textbf{3a: Duplicate question detected:} The question is discarded and not saved.
\end{itemize} \\
\hline
\end{tabular}

\paragraph{FR-AI02 - Suggest Personalized Quizzes}
The AI uses learning models (such as IRT) to analyze each student’s ability and suggest suitable quizzes to optimize their learning process.

\noindent
\begin{tabular}{|p{3cm}|p{13cm}|}
\hline
\textbf{Component} & \textbf{Description} \\
\hline
Requirements & Suggest Personalized Quizzes \\
\hline
Code & FR-AI02 \\
\hline
Primary Actor & \textbf{AI Engine} \\
\hline
Goal & Analyze each student’s ability based on their quiz history and apply a learning model (e.g., Item Response Theory - IRT) to recommend quizzes with adaptive difficulty, optimizing their learning experience. \\
\hline
Preconditions & The system has historical data on scores, completion times, and the AI Engine is functioning normally. \\
\hline
Postconditions & The system displays a personalized list of recommended quizzes for the student. \\
\hline
Main Flow & \vspace{-0.3cm}
\begin{enumerate}[nosep, leftmargin=*]
    \item The AI collects and analyzes the student’s historical performance data.
    \item The AI estimates the student’s ability using the IRT model or an equivalent method.
    \item The AI selects quizzes with corresponding difficulty levels and relevant topics.
    \item The system displays the personalized quiz recommendations to the student.
\end{enumerate} \\
\hline
Alternative Flow & \vspace{-0.3cm}
\begin{itemize}[nosep, leftmargin=*]
    \item \textbf{1a: Missing student data:} The system displays “Insufficient data to suggest quizzes.”
    \item \textbf{3a: Model error or no suitable quiz found:} The system displays “No suitable quizzes available, please try again later.”
\end{itemize} \\
\hline
\end{tabular}
\subsection{Non-Functional Requirements (NFR)}

\paragraph{NFR-01 - Usability}
The user interface must be friendly, intuitive, and easy to use for both students (teenagers) and teachers.

\paragraph{NFR-02 - Performance}
The system must maintain a response time of under 2 seconds for quiz interactions and submissions, even during peak load conditions.

\paragraph{NFR-03 - Security}
User personal data, including login credentials and scores, must be securely encrypted and protected from unauthorized access (e.g., using SSL/TLS and password hashing).

\paragraph{NFR-04 - Availability}
The system must operate reliably 24/7, ensuring that monthly downtime remains below 1\%.

\paragraph{NFR-05 - Scalability}
The system must be built on an architecture that supports easy scalability (both horizontal and vertical scaling) to handle increased user loads, especially during exam periods.

