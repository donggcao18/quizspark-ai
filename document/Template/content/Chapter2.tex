\chapter{SYSTEM DESIGN}
\section{Use-case Diagram}

QuizSpark supports three primary actors: \textit{Teacher}, \textit{Student}, and \textit{Admin}, each representing a distinct role with specific responsibilities and access permissions.

\begin{figure}[H]
    \centering
    \includegraphics[width=0.9\textwidth]{images/UseCase}
    \caption{Use Case Diagram of the QuizSpark System}
    \label{fig:usecase}
\end{figure}



\paragraph{Teacher}
The Teacher is responsible for organizing and managing learning activities. The main use cases associated with this actor include:
\begin{itemize}
    \item \textbf{Host Quiz Session}: Organize real-time quiz sessions for students.
    \item \textbf{Manage Classes}: Create and manage classes, add or remove students, and assign quizzes.
    \item \textbf{Auto-generate Questions (AI)}: Use AI to automatically generate quiz questions based on topic and difficulty.
    \item \textbf{Share Quizzes}: Share quizzes with students or other teachers.
    \item \textbf{Access Platform (Web/Mobile)}: Access the system via web or mobile platforms.
\end{itemize}

\paragraph{Student}
The Student is the primary learner using the system to participate in quizzes and review learning progress. The main use cases include:
\begin{itemize}
    \item \textbf{Join Quiz Session}: Join a live quiz session using a session code or link.
    \item \textbf{Submit Quiz \& View Results}: Submit completed quizzes and view scores and detailed feedback.
    \item \textbf{View Suggested Quizzes}: View quizzes recommended by the system.
    \item \textbf{Suggest Personalized Quizzes (AI)}: Receive personalized quiz recommendations generated by the AI engine.
    \item \textbf{Access Platform (Web/Mobile)}: Use the system on different platforms with synchronized data.
\end{itemize}

The use case \textit{Submit Quiz \& View Results} makes use of \textit{Suggest Personalized Quizzes (AI)} to support adaptive and personalized learning.

\paragraph{Admin}
The Admin manages the overall system operation and content quality. The main use cases include:
\begin{itemize}
    \item \textbf{Manage Users (Add/Block/Delete)}: Manage user accounts and permissions.
    \item \textbf{Moderate Content}: Review and moderate quizzes and user-generated content.
    \item \textbf{Share Quizzes}: Control or manage shared quizzes within the platform.
    \item \textbf{Access Platform (Web/Mobile)}: Access administrative features of the system.
\end{itemize}

The diagram includes the following relationships:
\begin{itemize}
    \item \textbf{<<uses>>}: Indicates that a use case relies on another use case, especially for AI-supported functionalities.
    \item \textbf{General functions}: Common system access through \textit{Access Platform (Web/Mobile)} shared among actors.
\end{itemize}

\section{Component Diagram}

The Component Diagram illustrates the high-level architecture of the Quiz System by describing its main components and the interactions among them. The system follows a layered and modular design, where each component has a well-defined responsibility, ensuring scalability, maintainability, and ease of integration.

\begin{figure}[H]
    \centering
    \includegraphics[width=0.8\textwidth]{images/Component.png}
    \caption{QuizSpark component diagram}
    \label{fig:component_dig}
\end{figure}

\subsection{Web Application Module}

\textbf{Function:}  
The Web Application Module provides the user interface and serves as the main interaction point between end users and the system. It is responsible for receiving client requests and presenting system responses in a user-friendly manner.

\textbf{Interactions:}
\begin{itemize}
    \item Sends requests such as quiz creation, quiz execution, result viewing, and user management to the Quiz System Core.
    \item Receives responses from the Core and displays the processed results to users.
\end{itemize}

\textbf{Technology:}  
Web technologies, including React, HTML, CSS, and JavaScript.

\subsection{Quiz System Core}

\textbf{Function:}  
The Quiz System Core acts as the central component of the system and is responsible for handling all core business logic, including:
\begin{itemize}
    \item Creating, managing, and executing quizzes.
    \item Managing quiz session states and evaluation processes.
    \item Coordinating communication between internal and external components.
\end{itemize}

\textbf{Interactions:}
\begin{itemize}
    \item Receives requests from the Web Application Module.
    \item Invokes the Data Access component to query and update the Quiz DB and User DB.
    \item Sends requests to the AI Generator for automatic question generation or personalized quiz recommendations.
    \item Interacts with the Template Generator and PDF Converter when quizzes are generated from templates or documents.
\end{itemize}

\textbf{Technology:}  
Backend implemented using Java Spring Boot.

\subsection{Data Access}

\textbf{Function:}  
The Data Access component is responsible for persistent data management and serves as an abstraction layer between the Quiz System Core and the underlying databases.

\textbf{Interactions:}
\begin{itemize}
    \item Executes SQL queries on the Quiz DB to manage quizzes, questions, sessions, and results.
    \item Executes SQL queries on the User DB to manage user profiles, roles, and access permissions.
\end{itemize}

\textbf{Notes:}  
This component must ensure efficient query execution, data security, and transaction integrity.

\subsection{AI Generator}

\textbf{Function:}  
The AI Generator provides intelligent services to enhance the functionality of the Quiz System, including:
\begin{itemize}
    \item Automatic generation of questions based on topic, difficulty level, and question format.
    \item Personalized quiz recommendations derived from students' learning history using models such as Item Response Theory (IRT).
\end{itemize}

\textbf{Interactions:}
\begin{itemize}
    \item Receives requests from the Quiz System Core.
    \item Returns generated questions or recommendations to the Core for storage or presentation.
\end{itemize}

\textbf{Technology:}  
A Python Flask-based microservice integrated with machine learning models.

\subsection{Template Generator and PDF Converter}

\textbf{Function:}  
This component handles template processing and document conversion tasks, including:
\begin{itemize}
    \item Extracting structured data from PDF documents.
    \item Creating and managing templates for quizzes and questions.
\end{itemize}

\textbf{Interactions:}
\begin{itemize}
    \item Receives input data from PDF files.
    \item Is invoked by the Quiz System Core when quizzes need to be generated from document-based sources.
\end{itemize}

\subsection{Database}

\textbf{Quiz DB:}
\begin{itemize}
    \item Stores quizzes, quiz sessions, questions, and evaluation results.
\end{itemize}

\textbf{User DB:}
\begin{itemize}
    \item Stores user information, access rights, and role-based permissions.
\end{itemize}

\subsection{Typical Interaction Flow}

\begin{enumerate}
    \item A user interacts with the Web Application Module.
    \item The Web Application Module sends a request to the Quiz System Core.
    \item The Core processes the request and may invoke:
    \begin{itemize}
        \item Data Access to retrieve or update data from the databases.
        \item AI Generator to generate questions or personalized recommendations.
        \item Template Generator and PDF Converter to process templates or PDF documents.
    \end{itemize}
    \item The Quiz System Core returns a response to the Web Application Module.
    \item The Web Application Module presents the final result to the user.
\end{enumerate}


\section{Sequence Diagrams}

This section describes two key interaction flows of the QuizSpark system using sequence diagrams: 
\\
(1) the process of a student joining and completing a quiz session (FR-S01)
\\
(2) the process of automatically generating questions using AI (FR-AI01).

\subsection{FR-S01 -- Join Quiz Session}

The sequence diagram for FR-S01 illustrates the real-time interaction flow when a student participates in a quiz session. 
The main objective is to describe the process from entering a quiz code to receiving the final result.

\begin{figure}[H]
    \centering
    \includegraphics[width=0.95\textwidth]{images/FR-S01}
    \caption{FR-S01 -- Join Quiz Session Sequence Diagram}
    \label{fig:fr-s01}
\end{figure}

\paragraph{Main Flow}
\begin{itemize}
    \item The student enters a quiz code and submits a join request via the Web/Mobile UI.
    \item The UI sends the request (quizCode, userId) to the Spring Boot server.
    \item The server validates the quiz code and user information using the database.
    \item If valid, the server returns quiz metadata and session status.
    \item If required, the server requests a randomized question set from the AI service.
    \item The AI service generates questions and returns them to the server.
    \item The server sends the quiz content and start signal to the client.
    \item The student answers questions and submits responses.
    \item The server stores answers, calculates the score, and saves results to the database.
    \item The final score and feedback are returned and displayed to the student.
\end{itemize}

\paragraph{Alternative Flow}
\begin{itemize}
    \item If the quiz code is invalid or expired, the server returns an error message.
    \item The UI displays the corresponding error to the student.
\end{itemize}

This sequence represents a real-time interaction flow with conditional branching, suitable for competitive quiz sessions.

---

\subsection{FR-AI01 -- Generate Questions (AI)}

The sequence diagram for FR-AI01 describes the process of automatically generating quiz questions using an AI service based on user-defined parameters.

\begin{figure}[H]
    \centering
    \includegraphics[width=0.95\textwidth]{images/FR-AI01}
    \caption{FR-AI01 -- Generate Questions (AI) Sequence Diagram}
    \label{fig:fr-ai01}
\end{figure}

\paragraph{Main Flow}
\begin{itemize}
    \item The user (Teacher or Admin) inputs the topic, difficulty level, and question format.
    \item The Web/Mobile UI sends a generation request to the Spring Boot server.
    \item The server forwards the request to the AI service (Flask).
    \item The AI service uses a machine learning model to generate questions.
    \item Generated questions are returned to the server.
    \item The server stores the generated questions in the database.
    \item The generated questions are sent back to the client.
    \item The UI displays the questions for review and optional editing.
\end{itemize}

This sequence represents a batch-style processing flow, where AI is used as an independent microservice to enhance quiz creation efficiency.




\section{Database Design}

\subsection{Entity Relationship (ER) Diagram}

The database of the QuizSpark system is designed to support quiz management, user management, real-time quiz sessions, and AI-powered functionalities. 
The Entity Relationship (ER) Diagram illustrates the core entities, their attributes, and the relationships among them.

\begin{figure}[H]
    \centering
    \includegraphics[width=\textwidth]{images/ER}
    \caption{Entity Relationship Diagram of the QuizSpark System}
    \label{fig:er-diagram}
\end{figure}

\subsection{Main Entities}

\paragraph{User}
The \textit{User} entity represents all system users, including Students, Teachers, and Admins.
It stores authentication and role-related information and serves as the base entity for access control.

\paragraph{Quiz}
The \textit{Quiz} entity stores information about quizzes created by users, including title, description, visibility, and ownership.
Each quiz is associated with one creator and contains multiple questions.

\paragraph{Question}
The \textit{Question} entity represents individual questions belonging to a quiz.
It stores question content, type, difficulty level, correct answers, and optional AI-generated metadata.

\paragraph{Quiz Session}
The \textit{Quiz Session} entity manages real-time quiz activities.
It links a quiz to a live session identified by a session code and tracks session status and timing.

\paragraph{Submission}
The \textit{Submission} entity records students’ quiz attempts.
It stores selected answers, submission time, and final scores, enabling result analysis and adaptive learning.

\paragraph{Class}
The \textit{Class} entity supports classroom management features.
Teachers can create classes, enroll students, and assign quizzes to specific classes.

---

\subsection{Relationships}

The main relationships between entities include:
\begin{itemize}
    \item A \textit{User} can create multiple \textit{Quizzes}, but each quiz is created by exactly one user.
    \item A \textit{Quiz} consists of multiple \textit{Questions}.
    \item A \textit{Quiz} can be hosted in multiple \textit{Quiz Sessions}.
    \item A \textit{Student} can submit multiple \textit{Submissions}, each linked to a specific quiz session.
    \item A \textit{Teacher} can manage multiple \textit{Classes}, and each class can contain multiple students.
\end{itemize}

