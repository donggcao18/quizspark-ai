\chapter{Requirements Analysis}

\section{Current System Analysis}

\subsection{User and Stakeholder Survey}
% Findings from users and stakeholders
Analyzing stakeholders is essential to ensure that \textbf{QuizSpark} effectively meets the needs of all user groups and system operators.

\begin{table}[h]
\centering
\begin{tabular}{|p{4cm}|p{6cm}|p{6cm}|}
\hline
\textbf{Stakeholder} & \textbf{Role / Objective} & \textbf{Main Needs} \\
\hline
Students / Learners & Primary users of the system. Their goal is to improve academic performance through personalized learning paths. & Take adaptive quizzes, track detailed learning progress, and receive study suggestions based on weaknesses.\\
\hline
Teachers / Tutors & Content creators and managers. Their goal is to enhance teaching and assessment effectiveness. & Manage classes, create high-quality exams, and view student statistics and class performance analytics.\\
\hline
Developers & Build and deploy system features. Their goal is performance and reliability. & Design scalable, maintainable code and integrate AI/ML technologies.\\
\hline
System Administrators (Admins) & Manage users and content across the platform. Their goal is safety and compliance. & Moderate content, secure user data, and monitor system performance and stability.\\
\hline
Advertising Partners & Provide revenue for the platform. Their goal is to reach appropriate target audiences. & Implement contextual advertising mechanisms suitable for students, ensuring ads are non-intrusive and do not disrupt the learning experience.\\
\hline
\end{tabular}
\caption{Stakeholders of the QuizSpark system and their main needs}
\label{tab:stakeholder}
\end{table}
So, the need of \textbf{Students} (Adaptive Testing, Detailed Progress Tracking) are central, directly influencing Project Objective . The needs of \textbf{Teachers} (Content Creation, Analytics) determine the usefulness of the tool in real educational environments.

\subsection{Review of Existing Systems}
To better understand the context and identify gaps in current learning platforms, several popular online assessment systems were reviewed, including \textbf{Quizizz}, \textbf{Azota}, and \textbf{Google Forms}. These systems are widely used in educational environments and provide a solid baseline for comparison.

\begin{itemize}
    \item \textbf{Quizizz:} Provides an engaging, gamified quiz experience with real-time participation and leaderboards. However, it mainly focuses on entertainment and lacks advanced personalization mechanisms or deep learning analytics.
    
    \item \textbf{Azota:} Commonly used for online exams and homework management. While it supports automatic grading and exam organization, it offers limited adaptive learning features and minimal post-quiz feedback beyond scores.
    
    \item \textbf{Google Forms:} A flexible tool for creating quizzes and surveys. Despite its simplicity and ease of use, it does not support real-time competition, adaptive difficulty, or intelligent feedback for learners.
\end{itemize}

Based on this review, existing systems primarily emphasize quiz delivery and basic assessment, but they do not adequately support adaptive learning, personalized feedback, or intelligent content generation. These limitations highlight the need for a more learner-centered and intelligent assessment platform.

\subsection{Key Features to Be Developed}
Based on the stakeholder survey and the review of existing systems, the following core features are identified as essential for the QuizSpark system:

\begin{itemize}
    \item \textbf{Adaptive Quiz Mechanism:} Automatically adjust question difficulty based on learner performance to provide personalized learning paths.
    
    \item \textbf{AI-based Question Generation:} Support teachers and students by automatically generating high-quality questions based on topic, difficulty level, and format.
    
    \item \textbf{Detailed Learning Analytics:} Provide in-depth feedback, including performance analysis by topic, common mistakes, and progress over time.
    
    \item \textbf{Real-time Competitive Quizzes:} Enable live quiz sessions with leaderboards to enhance motivation and engagement among students.
    
    \item \textbf{Multi-platform Accessibility:} Ensure seamless access across web and mobile platforms with synchronized data.
    
    \item \textbf{Content Management and Moderation:} Allow administrators to manage users and moderate content to maintain quality and system reliability.
\end{itemize}

These features collectively aim to address the shortcomings of existing systems while fulfilling the learning needs of students and instructional requirements of teachers.


\section{Functional Overview}
% High-level functional description

\subsection{Overall Use Case Diagram}
This overall use case diagram (Figure~\ref{fig:usecase_overall}) summarizes the core interactions in QuizSpark. 
% Its main actors are \textbf{Students}, \textbf{Teachers}, and supporting \textbf{AI services}. Students take quizzes, join live sessions, submit answers, and review their learning progress. Teachers create and manage quizzes, evaluate question quality, host live sessions, manage classes, and analyze performance results. AI services assist both Students and Teachers by generating questions and recommending personalized quizzes, thereby enabling efficient content creation and adaptive learning.
\textbf{-- Main Actors}
\begin{itemize}
    \item \textbf{Student}: Learns by taking quizzes, joining live sessions, and reviewing progress.
    \item \textbf{Teacher}: Designs and manages quizzes, classes, and live quiz sessions.
    \item \textbf{AI Service}: Supports both roles with automatic question generation and adaptive recommendations.
\end{itemize}

\textbf{-- Main Use Cases}
\begin{itemize}
    \item \textbf{Take / Practice Quiz}: Student answers questions in self-paced quizzes.
    \item \textbf{Join Live Session}: Student participates in teacher-hosted, real-time quiz sessions.
    \item \textbf{View Results \& Progress}: Student reviews scores, history, and feedback.
    \item \textbf{Create \& Manage Quizzes}: Teacher builds, edits, and organizes quiz content.
    \item \textbf{Manage Classes}: Teacher manages class lists and assigns quizzes to groups of students.
    \item \textbf{Analyze Performance}: Teacher views analytics for classes and individual students.
    \item \textbf{AI Question Generation}: AI service generates quiz questions from topics/documents.
    \item \textbf{Adaptive Quiz Recommendation}: AI service recommends or adapts quizzes based on student history.
\end{itemize}

\begin{figure}[h]
    \centering
    \includegraphics[width=\textwidth]{images/UseCase.png}
    \caption{Overall Use Case Diagram of the QuizSpark System}
    \label{fig:usecase_overall}
\end{figure}


\subsection{Detailed Use Case Diagrams}
% Decomposed use case diagrams

\begin{figure}[H]
    \centering
    \includegraphics[width=0.8\textwidth]{images/stuUC}
    \caption{Detailed Student Use Case Diagram}
    \label{fig:stuUC}
\end{figure}

\begin{figure}[H]
    \centering
    \includegraphics[width=0.8\textwidth]{images/tcUC}
    \caption{Detailed Teacher Use Case Diagram}
    \label{fig:tcUC}
\end{figure}

\subsection{Functional Specifications (Detailed Use Case Specifications)}

% --- GENERAL FUNCTIONS ---
\subsubsection{General Functions (Code: FR-G)}

\paragraph{FR-G01: Login / Register}
This service allows users (Students and Teachers) to access the QuizSpark system using registered credentials or by creating a new account.

\begin{table}[H]
\centering
\begin{tabular}{|p{3cm}|p{13cm}|}
\hline
\textbf{Component} & \textbf{Content} \\
\hline
Requirements & The system must provide an interface for authentication and new account registration. \\ 
\hline
Code & FR-G01 \\
\hline
Main Actor & Student, Teacher \\
\hline
Goal & To provide secure access to the system. \\
\hline
Precondition & Device is connected to the Internet. \\
\hline
Postcondition & User is authenticated and redirected to their personal Dashboard. \\
\hline
Main Flow & \vspace{-0.3cm}
\begin{enumerate}[nosep, leftmargin=*]
    \item User accesses the homepage and selects "Login" or "Register".
    \item For Login: User enters Email/Username and Password. The system validates the data.
    \item For Registration: User enters personal details, selects a role, and sets a password.
    \item The system encrypts the information, saves it to the database, and sends a success notification.
    \item User is automatically logged into the system.
\end{enumerate} \\
\hline
Alternative Flow & \vspace{-0.3cm}
\begin{itemize}[nosep, leftmargin=*]
    \item \textbf{2a: Invalid credentials:} System displays "Invalid credentials" error and prompts for re-entry.
    \item \textbf{3a: Email already exists:} System requests the user to use a different email or recover the password.
\end{itemize} \\
\hline
\end{tabular}
\caption{Functional specification for FR-G01: Login / Register}
\label{tab:fr-g01-login-register}
\end{table}

\paragraph{FR-G02: Manage Profile}
Allows users to manage personal information, update profile pictures, and change passwords periodically.

\begin{table}[H]
\centering
\begin{tabular}{|p{3cm}|p{13cm}|}
\hline
\textbf{Component} & \textbf{Content} \\
\hline
Requirements & Allow viewing and editing of basic personal profile fields. \\ 
\hline
Code & FR-G02 \\
\hline
Main Actor & Student, Teacher \\
\hline
Goal & To maintain accurate user information on the system. \\
\hline
Precondition & User is successfully logged in. \\
\hline
Postcondition & New information is updated in the database. \\
\hline
Main Flow & \vspace{-0.3cm}
\begin{enumerate}[nosep, leftmargin=*]
    \item User navigates to the "Profile Settings" section.
    \item User edits desired fields (Display Name, Avatar, Bio, etc.).
    \item User clicks "Save Changes".
    \item The system validates the data integrity and updates the database.
    \item A success notification is displayed.
\end{enumerate} \\
\hline
\end{tabular}
\caption{Functional specification for FR-G02: Manage Profile}
\label{tab:fr-g02-manage-profile}
\end{table}

% --- STUDENT FUNCTIONS ---
\subsubsection{Student Functions (Code: FR-U)}

\paragraph{FR-U01: Take Practice Quiz}
Allows students to practice independently with available question sets to reinforce knowledge.

\begin{table}[H]
\centering
\begin{tabular}{|p{3cm}|p{13cm}|}
\hline
\textbf{Component} & \textbf{Content} \\
\hline
Requirements & Display the list of questions, record answers, and provide instant scoring. \\ 
\hline
Code & FR-U01 \\
\hline
Main Actor & Student \\
\hline
Goal & To self-assess personal capacity through sample tests. \\
\hline
Precondition & Student is logged in. \\
\hline
Postcondition & Quiz results are saved in the student's learning history. \\
\hline
Main Flow & \vspace{-0.3cm}
\begin{enumerate}[nosep, leftmargin=*]
    \item Student searches for or selects a quiz set from the library.
    \item Student clicks "Start Quiz".
    \item System displays each question along with options (FR-U01.1).
    \item Student completes the questions and clicks "Submit".
    \item System cross-checks answers and displays the final score (FR-U02).
\end{enumerate} \\
\hline
\end{tabular}
\caption{Functional specification for FR-U01: Take Practice Quiz}
\label{tab:fr-u01-take-practice-quiz}
\end{table}

\paragraph{FR-U02: Submit \& View Results}
The system automatically grades the quiz and provides detailed feedback to the student.

\begin{table}[H]
\centering
\begin{tabular}{|p{3cm}|p{13cm}|}
\hline
\textbf{Component} & \textbf{Content} \\
\hline
Requirements & Calculate scores based on question weights and provide answer explanations. \\ 
\hline
Code & FR-U02 \\
\hline
Main Actor & Student \\
\hline
Goal & To receive evaluation results immediately after completing the test. \\
\hline
Precondition & Currently taking a quiz (FR-U01) or participating in a Live Session. \\
\hline
Postcondition & A result report is generated and stored. \\
\hline
Main Flow & \vspace{-0.3cm}
\begin{enumerate}[nosep, leftmargin=*]
    \item System receives answer data when the user clicks "Submit".
    \item Grading algorithm matches answers with the key and calculates the total score.
    \item System displays the result screen: Score, correct/incorrect count, and completion time.
    \item Student is allowed to review detailed questions and explanations.
\end{enumerate} \\
\hline
\end{tabular}
\caption{Functional specification for FR-U02: Submit and View Results}
\label{tab:fr-u02-submit-view-results}
\end{table}

\paragraph{FR-U03: Join Live Session}
Participate in real-time quiz sessions organized by teachers.

\begin{table}[H]
\centering
\begin{tabular}{|p{3cm}|p{13cm}|}
\hline
\textbf{Component} & \textbf{Content} \\
\hline
Requirements & Synchronized connection between teacher and student screens via WebSockets. \\ 
\hline
Code & FR-U03 \\
\hline
Main Actor & Student \\
\hline
Goal & To interact directly and compete with classmates in real-time. \\
\hline
Precondition & Student has a Game PIN or an access link provided by the Teacher. \\
\hline
Postcondition & Student's name appears on the class Leaderboard. \\
\hline
Main Flow & \vspace{-0.3cm}
\begin{enumerate}[nosep, leftmargin=*]
    \item Student enters the provided Game PIN.
    \item Student waits in the "Waiting Room" until the Teacher clicks Start.
    \item Student answers questions appearing on the screen within the time limit.
    \item Student views their personal rank on the leaderboard after each question.
\end{enumerate} \\
\hline
\end{tabular}
\caption{Functional specification for FR-U03: Join Live Session}
\label{tab:fr-u03-join-live-session}
\end{table}

\paragraph{FR-U04: AI Learning Recommendations}
Uses Artificial Intelligence to suggest learning paths and quiz sets tailored to the student's ability.

\begin{table}[H]
\centering
\begin{tabular}{|p{3cm}|p{13cm}|}
\hline
\textbf{Component} & \textbf{Content} \\
\hline
Requirements & Analyze historical quiz data to identify knowledge gaps. \\ 
\hline
Code & FR-U04 \\
\hline
Main Actor & Student \\
\hline
Goal & To personalize the learning experience and improve weak areas. \\
\hline
Precondition & At least 3 quiz attempts are recorded in history. \\
\hline
Postcondition & A list of recommendations is updated on the Dashboard. \\
\hline
Main Flow & \vspace{-0.3cm}
\begin{enumerate}[nosep, leftmargin=*]
    \item AI system scans quiz history and identifies topics with high error rates.
    \item AI calculates the student's current proficiency level.
    \item AI generates suggestions: "Topics to Improve", "Recommended for You".
    \item Student selects and participates based on the suggestions.
\end{enumerate} \\
\hline
\end{tabular}
\caption{Functional specification for FR-U04: AI Learning Recommendations}
\label{tab:fr-u04-ai-learning-recommendations}
\end{table}

% --- TEACHER FUNCTIONS ---
\subsubsection{Teacher Functions (Code: FR-T)}

\paragraph{FR-T01: Create \& Manage Quizzes}
Provides tools to author various question types (Multiple choice, Essay, Matching).

\begin{table}[H]
\centering
\begin{tabular}{|p{3cm}|p{13cm}|}
\hline
\textbf{Component} & \textbf{Content} \\
\hline
Requirements & Intuitive editor supporting image attachments and mathematical formulas. \\ 
\hline
Code & FR-T01 \\
\hline
Main Actor & Teacher \\
\hline
Goal & To build a digital learning resource repository for teaching. \\
\hline
Precondition & Teacher is logged in. \\
\hline
Postcondition & Quiz is published and can be shared publicly or internally. \\
\hline
Main Flow & \vspace{-0.3cm}
\begin{enumerate}[nosep, leftmargin=*]
    \item Teacher selects "Create New Quiz".
    \item Teacher sets general info (Title, Description, Subject, Time limit).
    \item Teacher adds questions manually or uses AI suggestions (FR-T04).
    \item Teacher sets correct answers and point weights.
    \item Teacher clicks "Publish" to finalize.
\end{enumerate} \\
\hline
\end{tabular}
\caption{Functional specification for FR-T01: Create and Manage Quizzes}
\label{tab:fr-t01-create-manage-quizzes}
\end{table}

\paragraph{FR-T02: Manage Classes \& Students}
Organize students into classes for easy assignment distribution and progress tracking.

\begin{table}[H]
\centering
\begin{tabular}{|p{3cm}|p{13cm}|}
\hline
\textbf{Component} & \textbf{Content} \\
\hline
Requirements & Management of class lists, Class Codes, and join invitations. \\ 
\hline
Code & FR-T02 \\
\hline
Main Actor & Teacher \\
\hline
Goal & Centralized management of student data by class unit. \\
\hline
Precondition & Teacher has a list of student emails or identifiers. \\
\hline
Postcondition & Class structure is established on the system. \\
\hline
Main Flow & \vspace{-0.3cm}
\begin{enumerate}[nosep, leftmargin=*]
    \item Teacher creates a new class and names it.
    \item System generates a unique "Class Code".
    \item Teacher sends the code to students or adds them manually via email.
    \item Teacher approves student join requests.
\end{enumerate} \\
\hline
\end{tabular}
\caption{Functional specification for FR-T02: Manage Classes and Students}
\label{tab:fr-t02-manage-classes-students}
\end{table}

\paragraph{FR-T03: View Class Analytics}
Provides visual statistical reports on the learning outcomes of the entire class and individuals.

\begin{table}[H]
\centering
\begin{tabular}{|p{3cm}|p{13cm}|}
\hline
\textbf{Component} & \textbf{Content} \\
\hline
Requirements & Bar/Pie charts showing score distribution and correct answer rates per question. \\ 
\hline
Code & FR-T03 \\
\hline
Main Actor & Teacher \\
\hline
Goal & To evaluate teaching effectiveness and student learning outcomes. \\
\hline
Precondition & The class has completed at least one quiz. \\
\hline
Postcondition & Analytical dashboards are available; data can be exported to PDF/Excel. \\
\hline
Main Flow & \vspace{-0.3cm}
\begin{enumerate}[nosep, leftmargin=*]
    \item Teacher selects a specific class and quiz.
    \item System aggregates data and displays analysis charts.
    \item Teacher reviews "difficult questions" (high error rates).
    \item Teacher views detailed reports for individual students to provide support.
\end{enumerate} \\
\hline
\end{tabular}
\caption{Functional specification for FR-T03: View Class Analytics}
\label{tab:fr-t03-view-class-analytics}
\end{table}

\paragraph{FR-T04: AI-Assisted Question Generation}
Integrates Generative AI to help teachers quickly draft quizzes from text or topics.

\begin{table}[H]
\centering
\begin{tabular}{|p{3cm}|p{13cm}|}
\hline
\textbf{Component} & \textbf{Content} \\
\hline
Requirements & Ability to extract questions from uploaded documents or keywords. \\ 
\hline
Code & FR-T04 \\
\hline
Main Actor & Teacher \\
\hline
Goal & To save preparation time and diversify the question bank. \\
\hline
Precondition & Teacher is in the quiz creation interface (FR-T01). \\
\hline
Postcondition & AI-suggested questions are added to the quiz draft. \\
\hline
Main Flow & \vspace{-0.3cm}
\begin{enumerate}[nosep, leftmargin=*]
    \item Teacher uploads a document (PDF/Docx) or enters a topic.
    \item Teacher selects the number of questions and difficulty level.
    \item AI generates a list of questions with answers.
    \item Teacher reviews, edits, and confirms the questions for the official quiz.
\end{enumerate} \\
\hline
\end{tabular}
\caption{Functional specification for FR-T04: AI-Assisted Question Generation}
\label{tab:fr-t04-ai-assisted-question-generation}
\end{table}
\section{Business Process Modeling}

This section presents the main business processes of the QuizSpark system using activity diagrams, complemented by selected sequence diagrams. The activity diagrams provide a high-level view of workflows from the perspective of primary actors, while the sequence diagrams enrich these workflows by clarifying interaction order and responsibility distribution among system components.

\subsection{Student Workflow}

\begin{figure}[H]
\centering
\includegraphics[width=0.6\textwidth]{images/stuAc}
\caption{Student Workflow Activity Diagram}
\label{fig:student_workflow}
\end{figure}

The student workflow captures the overall learning-oriented behavior, including quiz participation, answer submission, and progress review. This abstraction focuses on typical interaction patterns rather than individual feature execution, allowing both practice and live quiz scenarios to be represented within a single coherent process.

\begin{figure}[H]
\centering
\includegraphics[width=0.75\textwidth]{images/seq1}
\caption{Sequence Diagram – Student Joining and Taking a Quiz}
\label{fig:seq_student_take_quiz}
\end{figure}

This sequence diagram complements the activity workflow by detailing the interaction flow when a student joins a quiz and submits answers, highlighting coordination between the user interface, quiz handling logic, and session services.

\begin{figure}[H]
\centering
\includegraphics[width=0.75\textwidth]{images/seq2}
\caption{Sequence Diagram – Student Submitting Quiz and Viewing Results}
\label{fig:seq_student_results}
\end{figure}

The second sequence diagram refines the outcome phase of the workflow by illustrating how quiz submission triggers scoring, analytics processing, and AI-based recommendations, supporting progress tracking and personalized learning feedback.

\subsection{Teacher Workflow}

\begin{figure}[H]
\centering
\includegraphics[width=0.6\textwidth]{images/tcAc}
\caption{Teacher Workflow Activity Diagram}
\label{fig:teacher_workflow}
\end{figure}

The teacher workflow models instructional processes such as quiz preparation, optional AI-assisted support, session hosting, and performance evaluation. The emphasis is placed on decision points and optional paths rather than low-level operational details.

\begin{figure}[H]
\centering
\includegraphics[width=0.75\textwidth]{images/seq3}
\caption{Sequence Diagram – Teacher Creating and Managing Quizzes}
\label{fig:seq_teacher_create_quiz}
\end{figure}

This sequence diagram provides additional insight into quiz creation and management, clarifying how AI-based question generation is integrated as a supporting service within the teacher workflow.

\begin{figure}[H]
\centering
\includegraphics[width=0.75\textwidth]{images/seq4}
\caption{Sequence Diagram – Teacher Hosting a Live Quiz Session}
\label{fig:seq_teacher_host_session}
\end{figure}

The final sequence diagram complements the live session phase by illustrating interactions during quiz hosting, student participation, and real-time response handling, reinforcing the dynamic aspects of classroom-oriented activities.


\section{Non-Functional Requirements}
% System quality attributes

\subsection{Performance}

The performance requirements define the expected responsiveness and scalability of the system.

\begin{table}[H]
\centering
\begin{tabular}{|p{3cm}|p{7cm}|p{4cm}|}
\hline
\textbf{ID} & \textbf{Requirement} & \textbf{Metric / Constraint} \\
\hline
NFR-P01 & The system shall respond to user actions within an acceptable time. & Response time $\leq$ 2 seconds \\
\hline
NFR-P02 & The system shall support concurrent users during live quiz sessions. & $\geq$ 100 concurrent users \\
\hline
\end{tabular}
\caption{Non-functional performance requirements of QuizSpark}
\label{tab:nfr-performance}
\end{table}
\subsection{Security}

Security requirements ensure the protection of user data and system integrity.

\begin{table}[H]
\centering
\begin{tabular}{|p{3cm}|p{7cm}|p{4cm}|}
\hline
\textbf{ID} & \textbf{Requirement} & \textbf{Constraint} \\
\hline
NFR-S01 & User authentication shall be required for accessing protected features. & Role-based access control \\
\hline
NFR-S02 & Sensitive user data shall be securely stored and transmitted. & Encrypted communication (HTTPS) \\
\hline
\end{tabular}
\caption{Non-functional security requirements of QuizSpark}
\label{tab:nfr-security}
\end{table}
\subsection{Usability}

Usability requirements focus on ease of use and learning experience.

\begin{table}[H]
\centering
\begin{tabular}{|p{3cm}|p{7cm}|p{4cm}|}
\hline
\textbf{ID} & \textbf{Requirement} & \textbf{Evaluation Criteria} \\
\hline
NFR-U01 & The system shall provide an intuitive user interface. & Basic tasks completed without training \\
\hline
NFR-U02 & The system shall provide clear feedback during quiz participation. & Immediate visual or textual feedback \\
\hline
\end{tabular}
\caption{Non-functional usability requirements of QuizSpark}
\label{tab:nfr-usability}
\end{table}

\subsection{Maintainability}

Maintainability requirements describe how easily the system can be modified and extended.

\begin{table}[H]
\centering
\begin{tabular}{|p{3cm}|p{7cm}|p{4cm}|}
\hline
\textbf{ID} & \textbf{Requirement} & \textbf{Design Constraint} \\
\hline
NFR-M01 & The system shall be modular and loosely coupled. & Layered architecture \\
\hline
NFR-M02 & The system shall support future feature extensions. & Well-defined APIs and documentation \\
\hline
\end{tabular}
\caption{Non-functional maintainability requirements of QuizSpark}
\label{tab:nfr-maintainability}
\end{table}
