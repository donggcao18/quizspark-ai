\chapter{Requirements Analysis}

\section{Current System Analysis}

\subsection{User and Stakeholder Survey}
% Findings from users and stakeholders
Analyzing stakeholders is essential to ensure that \textbf{QuizSpark} effectively meets the needs of all user groups and system operators.

\begin{table}[h]
\centering
\begin{tabular}{|p{4cm}|p{6cm}|p{6cm}|}
\hline
\textbf{Stakeholder} & \textbf{Role / Objective} & \textbf{Main Needs} \\
\hline
Students / Learners & Primary users of the system. Their goal is to improve academic performance through personalized learning paths. & Take adaptive quizzes, track detailed learning progress, and receive study suggestions based on weaknesses.\\
\hline
Teachers / Tutors & Content creators and managers. Their goal is to enhance teaching and assessment effectiveness. & Manage classes, create high-quality exams, and view student statistics and class performance analytics.\\
\hline
Developers & Build and deploy system features. Their goal is performance and reliability. & Design scalable, maintainable code and integrate AI/ML technologies.\\
\hline
System Administrators (Admins) & Manage users and content across the platform. Their goal is safety and compliance. & Moderate content, secure user data, and monitor system performance and stability.\\
\hline
Advertising Partners & Provide revenue for the platform. Their goal is to reach appropriate target audiences. & Implement contextual advertising mechanisms suitable for students, ensuring ads are non-intrusive and do not disrupt the learning experience.\\
\hline
\end{tabular}
\caption{Stakeholders of the QuizSpark system and their main needs}
\label{tab:stakeholder}
\end{table}
So, the need of \textbf{Students} (Adaptive Testing, Detailed Progress Tracking) are central, directly influencing Project Objective . The needs of \textbf{Teachers} (Content Creation, Analytics) determine the usefulness of the tool in real educational environments.

\subsection{Review of Existing Systems}
To better understand the context and identify gaps in current learning platforms, several popular online assessment systems were reviewed, including \textbf{Quizizz}, \textbf{Azota}, and \textbf{Google Forms}. These systems are widely used in educational environments and provide a solid baseline for comparison.

\begin{itemize}
    \item \textbf{Quizizz:} Provides an engaging, gamified quiz experience with real-time participation and leaderboards. However, it mainly focuses on entertainment and lacks advanced personalization mechanisms or deep learning analytics.
    
    \item \textbf{Azota:} Commonly used for online exams and homework management. While it supports automatic grading and exam organization, it offers limited adaptive learning features and minimal post-quiz feedback beyond scores.
    
    \item \textbf{Google Forms:} A flexible tool for creating quizzes and surveys. Despite its simplicity and ease of use, it does not support real-time competition, adaptive difficulty, or intelligent feedback for learners.
\end{itemize}

Based on this review, existing systems primarily emphasize quiz delivery and basic assessment, but they do not adequately support adaptive learning, personalized feedback, or intelligent content generation. These limitations highlight the need for a more learner-centered and intelligent assessment platform.

\subsection{Key Features to Be Developed}
Based on the stakeholder survey and the review of existing systems, the following core features are identified as essential for the QuizSpark system:

\begin{itemize}
    \item \textbf{Adaptive Quiz Mechanism:} Automatically adjust question difficulty based on learner performance to provide personalized learning paths.
    
    \item \textbf{AI-based Question Generation:} Support teachers and students by automatically generating high-quality questions based on topic, difficulty level, and format.
    
    \item \textbf{Detailed Learning Analytics:} Provide in-depth feedback, including performance analysis by topic, common mistakes, and progress over time.
    
    \item \textbf{Real-time Competitive Quizzes:} Enable live quiz sessions with leaderboards to enhance motivation and engagement among students.
    
    \item \textbf{Multi-platform Accessibility:} Ensure seamless access across web and mobile platforms with synchronized data.
    
    \item \textbf{Content Management and Moderation:} Allow administrators to manage users and moderate content to maintain quality and system reliability.
\end{itemize}

These features collectively aim to address the shortcomings of existing systems while fulfilling the learning needs of students and instructional requirements of teachers.


\section{Functional Overview}
% High-level functional description

\subsection{Overall Use Case Diagram}
This overall use case diagram (Figure~\ref{fig:usecase_overall}) summarizes the core interactions in QuizSpark. 
% Its main actors are \textbf{Students}, \textbf{Teachers}, and supporting \textbf{AI services}. Students take quizzes, join live sessions, submit answers, and review their learning progress. Teachers create and manage quizzes, evaluate question quality, host live sessions, manage classes, and analyze performance results. AI services assist both Students and Teachers by generating questions and recommending personalized quizzes, thereby enabling efficient content creation and adaptive learning.
\textbf{-- Main Actors}
\begin{itemize}
    \item \textbf{Student}: Learns by taking quizzes, joining live sessions, and reviewing progress.
    \item \textbf{Teacher}: Designs and manages quizzes, classes, and live quiz sessions.
    \item \textbf{AI Service}: Supports both roles with automatic question generation and adaptive recommendations.
\end{itemize}

\textbf{-- Main Use Cases}
\begin{itemize}
    \item \textbf{Take / Practice Quiz}: Student answers questions in self-paced quizzes.
    \item \textbf{Join Live Session}: Student participates in teacher-hosted, real-time quiz sessions.
    \item \textbf{View Results \& Progress}: Student reviews scores, history, and feedback.
    \item \textbf{Create \& Manage Quizzes}: Teacher builds, edits, and organizes quiz content.
    \item \textbf{Manage Classes}: Teacher manages class lists and assigns quizzes to groups of students.
    \item \textbf{Analyze Performance}: Teacher views analytics for classes and individual students.
    \item \textbf{AI Question Generation}: AI service generates quiz questions from topics/documents.
    \item \textbf{Adaptive Quiz Recommendation}: AI service recommends or adapts quizzes based on student history.
\end{itemize}

\begin{figure}[h]
    \centering
    \includegraphics[width=\textwidth]{images/UseCase.png}
    \caption{Overall Use Case Diagram of the QuizSpark System}
    \label{fig:usecase_overall}
\end{figure}


\subsection{Detailed Use Case Diagrams}
% Decomposed use case diagrams

\begin{figure}[H]
    \centering
    \includegraphics[width=0.8\textwidth]{images/stuUC}
    \caption{Detailed Student Use Case Diagram}
    \label{fig:stuUC}
\end{figure}

\begin{figure}[H]
    \centering
    \includegraphics[width=0.8\textwidth]{images/tcUC}
    \caption{Detailed Teacher Use Case Diagram}
    \label{fig:tcUC}
\end{figure}

\subsection{Functional Specifications (Detailed Use Case Specifications)}

% --- GENERAL FUNCTIONS ---
\subsubsection{General Functions (Code: FR-G)}

\paragraph{FR-G01: Login / Register}
This service allows users (Students and Teachers) to access the QuizSpark system using registered credentials or by creating a new account.

\begin{table}[H]
\centering
\begin{tabular}{|p{3cm}|p{13cm}|}
\hline
\textbf{Component} & \textbf{Content} \\ \hline
Requirements & The system must provide an interface for authentication and new account registration. \\ \hline
Code & FR-G01 \\ \hline
Main Actor & Student, Teacher \\ \hline
Goal & To provide secure access to the system. \\ \hline
Precondition & Device is connected to the Internet. \\ \hline
Postcondition & User is authenticated and redirected to their personal Dashboard. \\ \hline
Main Flow & \vspace{-0.3cm}
\begin{enumerate}[nosep, leftmargin=*]
    \item User accesses the homepage and selects "Login" or "Register".
    \item For Login: User enters Email/Username and Password. The system validates the data.
    \item For Registration: User enters personal details, selects a role, and sets a password.
    \item The system encrypts the information, saves it to the database, and sends a success notification.
    \item User is automatically logged into the system.
\end{enumerate} \\ \hline
Alternative Flow & \vspace{-0.3cm}
\begin{itemize}[nosep, leftmargin=*]
    \item \textbf{2a: Invalid credentials:} System displays "Invalid credentials" error and prompts for re-entry.
    \item \textbf{3a: Email already exists:} System requests the user to use a different email or recover the password.
\end{itemize} \\ \hline
\end{tabular}
\end{table}

\paragraph{FR-G02: Create / Edit / Delete Personal Quizzes}
Allows users (both students and teachers) to manage their own quizzes for practice, learning, or sharing purposes.

\begin{table}[H]
\centering
\begin{tabular}{|p{3cm}|p{13cm}|}
\hline
Component & Content \\ \hline
Requirements & Users can create, modify, or delete quizzes, including questions and settings (private/public visibility). \\ \hline
Code & FR-G02 \\ \hline
Main Actor & Student, Teacher \\ \hline
Goal & Manage personal quiz content efficiently. \\ \hline
Precondition & User is logged in. \\ \hline
Postcondition & Quiz is stored, updated, or deleted in the database. \\ \hline
Main Flow & 
\begin{enumerate}[nosep, leftmargin=*]
    \item Navigate to "My Quizzes".
    \item Choose to create, edit, or delete a quiz.
    \item For creation: input title, description, visibility, and questions.
    \item Save changes; system validates and updates the database.
\end{enumerate} \\ \hline
Alternative Flow & 
\begin{itemize}[nosep, leftmargin=*]
    \item Missing required fields triggers notification.
    \item Invalid question format is rejected.
    \item Deletion canceled leaves quiz unchanged.
\end{itemize} \\ \hline
\end{tabular}
\end{table}

\paragraph{FR-G03: View Leaderboard}
Allows users to view leaderboards of scores and achievements within classes or system-wide.

\begin{table}[H]
\centering
\begin{tabular}{|p{3cm}|p{13cm}|}
\hline
Component & Content \\ \hline
Requirements & Display rankings to motivate learning and competition. \\ \hline
Code & FR-G03 \\ \hline
Main Actor & Student, Teacher \\ \hline
Goal & Encourage positive competition by showing performance. \\ \hline
Precondition & System contains quiz result data. \\ \hline
Postcondition & Leaderboard is displayed and can be filtered. \\ \hline
Main Flow &
\begin{enumerate}[nosep, leftmargin=*]
    \item User selects "Leaderboard".
    \item System retrieves quiz results.
    \item System sorts and displays rankings with names and scores.
\end{enumerate} \\ \hline
Alternative Flow &
\begin{itemize}[nosep, leftmargin=*]
    \item No quiz results: display "No leaderboard data available."
\end{itemize} \\ \hline
\end{tabular}
\end{table}

\paragraph{FR-G04: AI Question Generation}
Allows users to automatically generate questions using AI based on topic, difficulty, and type.

\begin{table}[H]
\centering
\begin{tabular}{|p{3cm}|p{13cm}|}
\hline
Component & Content \\ \hline
Requirements & Generate valid questions automatically to assist quiz creation. \\ \hline
Code & FR-G04 \\ \hline
Main Actor & Student, Teacher \\ \hline
Goal & Facilitate faster quiz creation and study preparation. \\ \hline
Precondition & User is logged in; AI service is operational. \\ \hline
Postcondition & Generated questions are displayed for review and can be saved into quizzes. \\ \hline
Main Flow &
\begin{enumerate}[nosep, leftmargin=*]
    \item Open "AI Question Generator".
    \item Enter topic, difficulty, type, and number of questions.
    \item AI processes input and generates questions.
    \item System displays questions for selection or editing.
    \item Save selected questions to quiz.
\end{enumerate} \\ \hline
Alternative Flow &
\begin{itemize}[nosep, leftmargin=*]
    \item Missing input triggers "Please fill in all fields."
    \item AI generation fails: "Unable to generate questions. Try again later."
\end{itemize} \\ \hline
\end{tabular}
\end{table}

\paragraph{FR-G05: Multi-Platform Access}
Enables users to access QuizSpark via web or mobile with synchronized data.

\begin{table}[H]
\centering
\begin{tabular}{|p{3cm}|p{13cm}|}
\hline
Component & Content \\ \hline
Requirements & Seamless access on multiple devices with consistent experience. \\ \hline
Code & FR-G05 \\ \hline
Main Actor & Student, Teacher \\ \hline
Goal & Ensure accessibility and data synchronization across platforms. \\ \hline
Precondition & Registered account and stable internet. \\ \hline
Postcondition & User can access and perform actions across devices; data synced. \\ \hline
Main Flow &
\begin{enumerate}[nosep, leftmargin=*]
    \item Open QuizSpark on browser or mobile app.
    \item Log in using credentials.
    \item System retrieves user data and adapts UI to device.
    \item Actions on one device are reflected on others.
\end{enumerate} \\ \hline
Alternative Flow &
\begin{itemize}[nosep, leftmargin=*]
    \item Unsupported device: notify user.
    \item Connection error: switch to offline mode if available.
\end{itemize} \\ \hline
\end{tabular}
\end{table}

\paragraph{FR-G06: Share Quizzes}
Allows users to share quizzes publicly or with selected users.

\begin{table}[H]
\centering
\begin{tabular}{|p{3cm}|p{13cm}|}
\hline
Component & Content \\ \hline
Requirements & Share quizzes via public links or controlled access. \\ \hline
Code & FR-G06 \\ \hline
Main Actor & Student, Teacher \\ \hline
Goal & Enable collaborative learning and sharing. \\ \hline
Precondition & User has at least one quiz created. \\ \hline
Postcondition & Quiz is accessible to selected audience. \\ \hline
Main Flow &
\begin{enumerate}[nosep, leftmargin=*]
    \item Open "My Quizzes" and select a quiz.
    \item Choose sharing option: Public or Selected Users.
    \item System generates link or sets access permissions.
    \item Recipients are notified or can access quiz.
\end{enumerate} \\ \hline
Alternative Flow &
\begin{itemize}[nosep, leftmargin=*]
    \item No quizzes available: system notifies user.
    \item Invalid recipient: prompt for re-entry.
    \item Permission error: restrict access and show error.
\end{itemize} \\ \hline
\end{tabular}
\end{table}


\subsubsection{Student Functions (Code: FR-U)}

\paragraph{FR-U01: Join Quiz Session}
Allows students to join real-time competitive quiz sessions hosted by teachers using a session code or invitation link.

\begin{table}[H]
\centering
\begin{tabular}{|p{3cm}|p{13cm}|}
\hline
\textbf{Component} & \textbf{Content} \\ \hline
Requirements & Join a competitive quiz session using a valid code or link. \\ \hline
Code & FR-U01 \\ \hline
Main Actor & Student \\ \hline
Goal & Participate in live quizzes and answer questions in real time. \\ \hline
Precondition & Student is logged in and possesses a valid session code or invitation link. \\ \hline
Postcondition & Student joins the live quiz session successfully and can submit answers. \\ \hline
Main Flow & \vspace{-0.3cm}
\begin{enumerate}[nosep, leftmargin=*]
    \item Student selects “Join Quiz” from the dashboard.
    \item Student enters the session code or clicks the invitation link.
    \item System verifies session validity and participant eligibility.
    \item Student enters the quiz lobby and waits for the host to start the session.
    \item Once started, questions are presented sequentially for the student to answer.
    \item System updates the student's real-time score after each question submission.
\end{enumerate} \\ \hline
Alternative Flow & \vspace{-0.3cm}
\begin{itemize}[nosep, leftmargin=*]
    \item \textbf{2a: Invalid or expired code:} System displays “Invalid session code. Please check and try again.”
    \item \textbf{3a: Session full or closed:} System displays “Unable to join. The session is full or has ended.”
\end{itemize} \\ \hline
\end{tabular}
\end{table}

\paragraph{FR-U02: Submit Quiz \& View Results}
Allows students to submit completed quizzes, automatically receive grading results, and review detailed performance analytics.

\begin{table}[H]
\centering
\begin{tabular}{|p{3cm}|p{13cm}|}
\hline
\textbf{Component} & \textbf{Content} \\ \hline
Requirements & Submit answers, automatically grade, store results, and display analytics. \\ \hline
Code & FR-U02 \\ \hline
Main Actor & Student \\ \hline
Goal & Receive immediate feedback and performance analytics after quiz completion. \\ \hline
Precondition & Student has completed all questions in a valid quiz session. \\ \hline
Postcondition & Quiz graded, results stored in the database, and detailed analytics displayed to the student. \\ \hline
Main Flow & \vspace{-0.3cm}
\begin{enumerate}[nosep, leftmargin=*]
    \item Student clicks “Submit Quiz” after completing all questions.
    \item System asks for submission confirmation.
    \item Upon confirmation, system collects answers, locks editing, and sends data to the server.
    \item System automatically grades the quiz and calculates scores.
    \item Results, including score, completion time, and detailed responses, are stored in the database.
    \item System displays a detailed result screen with:
    \begin{itemize}[nosep, leftmargin=*]
        \item Total score and completion time
        \item Correct/incorrect answers with explanations (if available)
        \item Performance charts summarizing strengths and weaknesses by topic
    \end{itemize}
    \item Student can download or review detailed reports.
\end{enumerate} \\ \hline
Alternative Flow & \vspace{-0.3cm}
\begin{itemize}[nosep, leftmargin=*]
    \item \textbf{2a: Submission canceled:} Student returns to the quiz interface.
    \item \textbf{3a: Connection lost:} Progress is saved temporarily and resubmitted automatically once restored.
    \item \textbf{4a: Grading error:} Quiz is marked “Pending Review” for teacher evaluation.
    \item \textbf{6a: Missing analytics data:} Display available results with a warning message.
\end{itemize} \\ \hline
\end{tabular}
\end{table}


\subsubsection{Teacher Functions (Code: FR-T)}

\paragraph{FR-T01: Manage Classes}
This function allows teachers to manage class lists, including adding or removing students and assigning quizzes to classes.

\begin{table}[H]
\centering
\begin{tabular}{|p{3cm}|p{13cm}|}
\hline
\textbf{Component} & \textbf{Content} \\ \hline
Requirements & Manage class lists, add/remove students, assign quizzes to classes. \\ \hline
Code & FR-T01 \\ \hline
Main Actor & Teacher \\ \hline
Goal & Organize classes effectively and track student learning progress. \\ \hline
Precondition & Teacher is logged in and has at least one class created. \\ \hline
Postcondition & Class list updated; students added/removed; quizzes assigned successfully. \\ \hline
Main Flow & \vspace{-0.3cm}
\begin{enumerate}[nosep, leftmargin=*]
    \item Teacher navigates to the "Class Management" section.
    \item System displays the list of existing classes.
    \item Teacher selects a class to edit.
    \item Teacher adds a new student, removes a student, or assigns a quiz to the class.
    \item System saves the changes and shows a confirmation message.
\end{enumerate} \\ \hline
Alternative Flow & \vspace{-0.3cm}
\begin{itemize}[nosep, leftmargin=*]
    \item \textbf{4a: Student not found:} System shows "Student not found" for invalid ID/email.
    \item \textbf{4b: Quiz assignment error:} System displays "Unable to assign quiz. Please try again later" if assignment fails.
\end{itemize} \\ \hline
\end{tabular}
\end{table}

\paragraph{FR-T02: Host Quiz Session}
Allows teachers to host and manage real-time competitive quiz sessions, inviting students via session code or link.

\begin{table}[H]
\centering
\begin{tabular}{|p{3cm}|p{13cm}|}
\hline
\textbf{Component} & \textbf{Content} \\ \hline
Requirements & Host real-time quiz sessions, monitor participation, view live results. \\ \hline
Code & FR-T02 \\ \hline
Main Actor & Teacher \\ \hline
Goal & Conduct interactive quizzes and monitor student engagement. \\ \hline
Precondition & Teacher is logged in and has at least one quiz ready to host. \\ \hline
Postcondition & Live session started; students can join using unique code/link; live results displayed. \\ \hline
Main Flow & \vspace{-0.3cm}
\begin{enumerate}[nosep, leftmargin=*]
    \item Teacher opens "Host Quiz" feature.
    \item Teacher selects a quiz to host.
    \item System generates a unique session code and invitation link.
    \item Teacher shares the code/link with students.
    \item Students join; teacher monitors participant status.
    \item Teacher starts the quiz; system presents questions to all participants.
    \item Teacher views live responses and leaderboard updates.
    \item After completion, system displays summary results.
\end{enumerate} \\ \hline
Alternative Flow & \vspace{-0.3cm}
\begin{itemize}[nosep, leftmargin=*]
    \item \textbf{2a: No quiz available:} System shows "Please create or select a quiz before hosting."
    \item \textbf{3a: Network issue or timeout:} System shows "Unable to start session. Please check your connection."
\end{itemize} \\ \hline
\end{tabular}
\end{table}

\section{Business Process Modeling}

This section presents the main business processes of the QuizSpark system using activity diagrams, complemented by selected sequence diagrams. The activity diagrams provide a high-level view of workflows from the perspective of primary actors, while the sequence diagrams enrich these workflows by clarifying interaction order and responsibility distribution among system components.

\subsection{Student Workflow}

\begin{figure}[H]
\centering
\includegraphics[width=0.6\textwidth]{images/stuAc}
\caption{Student Workflow Activity Diagram}
\label{fig:student_workflow}
\end{figure}

The student workflow captures the overall learning-oriented behavior, including quiz participation, answer submission, and progress review. This abstraction focuses on typical interaction patterns rather than individual feature execution, allowing both practice and live quiz scenarios to be represented within a single coherent process.

\begin{figure}[H]
\centering
\includegraphics[width=0.75\textwidth]{images/seq1}
\caption{Sequence Diagram – Student Joining and Taking a Quiz}
\label{fig:seq_student_take_quiz}
\end{figure}

This sequence diagram complements the activity workflow by detailing the interaction flow when a student joins a quiz and submits answers, highlighting coordination between the user interface, quiz handling logic, and session services.

\begin{figure}[H]
\centering
\includegraphics[width=0.75\textwidth]{images/seq2}
\caption{Sequence Diagram – Student Submitting Quiz and Viewing Results}
\label{fig:seq_student_results}
\end{figure}

The second sequence diagram refines the outcome phase of the workflow by illustrating how quiz submission triggers scoring, analytics processing, and AI-based recommendations, supporting progress tracking and personalized learning feedback.

\subsection{Teacher Workflow}

\begin{figure}[H]
\centering
\includegraphics[width=0.6\textwidth]{images/tcAc}
\caption{Teacher Workflow Activity Diagram}
\label{fig:teacher_workflow}
\end{figure}

The teacher workflow models instructional processes such as quiz preparation, optional AI-assisted support, session hosting, and performance evaluation. The emphasis is placed on decision points and optional paths rather than low-level operational details.

\begin{figure}[H]
\centering
\includegraphics[width=0.75\textwidth]{images/seq3}
\caption{Sequence Diagram – Teacher Creating and Managing Quizzes}
\label{fig:seq_teacher_create_quiz}
\end{figure}

This sequence diagram provides additional insight into quiz creation and management, clarifying how AI-based question generation is integrated as a supporting service within the teacher workflow.

\begin{figure}[H]
\centering
\includegraphics[width=0.75\textwidth]{images/seq4}
\caption{Sequence Diagram – Teacher Hosting a Live Quiz Session}
\label{fig:seq_teacher_host_session}
\end{figure}

The final sequence diagram complements the live session phase by illustrating interactions during quiz hosting, student participation, and real-time response handling, reinforcing the dynamic aspects of classroom-oriented activities.


\section{Non-Functional Requirements}
% System quality attributes

\subsection{Performance}

The performance requirements define the expected responsiveness and scalability of the system.

\begin{table}[H]
\centering
\begin{tabular}{|p{3cm}|p{7cm}|p{4cm}|}
\hline
\textbf{ID} & \textbf{Requirement} & \textbf{Metric / Constraint} \\
\hline
NFR-P01 & The system shall respond to user actions within an acceptable time. & Response time $\leq$ 2 seconds \\
\hline
NFR-P02 & The system shall support concurrent users during live quiz sessions. & $\geq$ 100 concurrent users \\
\hline
\end{tabular}
\caption{Non-functional performance requirements of QuizSpark}
\label{tab:nfr-performance}
\end{table}
\subsection{Security}

Security requirements ensure the protection of user data and system integrity.

\begin{table}[H]
\centering
\begin{tabular}{|p{3cm}|p{7cm}|p{4cm}|}
\hline
\textbf{ID} & \textbf{Requirement} & \textbf{Constraint} \\
\hline
NFR-S01 & User authentication shall be required for accessing protected features. & Role-based access control \\
\hline
NFR-S02 & Sensitive user data shall be securely stored and transmitted. & Encrypted communication (HTTPS) \\
\hline
\end{tabular}
\caption{Non-functional security requirements of QuizSpark}
\label{tab:nfr-security}
\end{table}
\subsection{Usability}

Usability requirements focus on ease of use and learning experience.

\begin{table}[H]
\centering
\begin{tabular}{|p{3cm}|p{7cm}|p{4cm}|}
\hline
\textbf{ID} & \textbf{Requirement} & \textbf{Evaluation Criteria} \\
\hline
NFR-U01 & The system shall provide an intuitive user interface. & Basic tasks completed without training \\
\hline
NFR-U02 & The system shall provide clear feedback during quiz participation. & Immediate visual or textual feedback \\
\hline
\end{tabular}
\caption{Non-functional usability requirements of QuizSpark}
\label{tab:nfr-usability}
\end{table}

\subsection{Maintainability}

Maintainability requirements describe how easily the system can be modified and extended.

\begin{table}[H]
\centering
\begin{tabular}{|p{3cm}|p{7cm}|p{4cm}|}
\hline
\textbf{ID} & \textbf{Requirement} & \textbf{Design Constraint} \\
\hline
NFR-M01 & The system shall be modular and loosely coupled. & Layered architecture \\
\hline
NFR-M02 & The system shall support future feature extensions. & Well-defined APIs and documentation \\
\hline
\end{tabular}
\caption{Non-functional maintainability requirements of QuizSpark}
\label{tab:nfr-maintainability}
\end{table}
