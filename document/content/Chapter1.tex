\chapter{Introduction}

\section{Problem Statement}

\subsection{Background and Motivation}

\begin{figure}[H]
    \centering
    \fbox{\parbox{0.9\textwidth}{
        \centering
        \vspace{2.2cm}
        \textit{Placeholder for comparison between traditional quiz systems and adaptive learning systems}
        \vspace{2.2cm}
    }}
    \caption{Overview of limitations in existing quiz platforms}
    \label{fig:problem_context}
\end{figure}

With the rapid growth of online education, quiz-based assessment systems have become an essential
tool for evaluating learner performance.
Platforms such as Quizizz, Google Forms, and similar systems are widely adopted due to their ease
of use and accessibility.
However, these systems primarily focus on static assessment rather than supporting continuous and
personalized learning.

\subsection{Identified Problems}

Through analysis of existing solutions, several key limitations have been identified:

\begin{itemize}
    \item \textbf{Lack of adaptive difficulty:}
    Most platforms provide a fixed set of questions, ignoring individual learner ability and progress.
    
    \item \textbf{Limited learning feedback:}
    Feedback is often restricted to numerical scores, without explaining learning gaps or misconceptions.
    
    \item \textbf{Manual content creation overhead:}
    Teachers must invest significant time in creating and maintaining large question banks.
\end{itemize}

These problems reduce learning effectiveness and limit the long-term educational value of quiz systems.

\section{Objectives and Scope}

\subsection{Project Objectives}

The QuizSpark project aims to address the identified problems by achieving the objectives listed in
Table~\ref{tab:intro_objectives}.

\begin{table}[H]
\centering
\caption{Objectives of the QuizSpark project}
\label{tab:intro_objectives}
\begin{tabular}{|p{4cm}|p{9cm}|}
\hline
\textbf{Objective} & \textbf{Description} \\ \hline
Adaptive assessment & Adjust quiz difficulty based on learner performance \\ \hline
AI-assisted content creation & Automatically generate quiz questions from topics or documents \\ \hline
Enhanced feedback & Provide detailed result analysis and learning insights \\ \hline
Interactive learning & Support real-time quiz sessions and leaderboards \\ \hline
\end{tabular}
\end{table}

\subsection{Project Scope}

\begin{figure}[H]
    \centering
    \fbox{\parbox{0.85\textwidth}{
        \centering
        \vspace{2cm}
        \textit{Placeholder for project scope diagram (in-scope vs out-of-scope)}
        \vspace{2cm}
    }}
    \caption{Scope definition of the QuizSpark project}
    \label{fig:project_scope}
\end{figure}

The scope of the project is defined as follows:

\begin{description}
    \item[In Scope:]
    Design and implementation of a web and mobile quiz system, AI-based question generation,
    adaptive learning mechanisms, and performance analytics.
    
    \item[Out of Scope:]
    Advanced AI model research, large-scale cloud optimization, and commercial deployment.
\end{description}

\section{Proposed Solution Approach}

\subsection{Overall Technical Direction}

\begin{figure}[H]
    \centering
    \fbox{\parbox{0.9\textwidth}{
        \centering
        \vspace{2.2cm}
        \textit{Placeholder for high-level system architecture diagram}
        \vspace{2.2cm}
    }}
    \caption{High-level architecture of the QuizSpark system}
    \label{fig:intro_architecture}
\end{figure}

To meet the project objectives, QuizSpark adopts a modular client-server architecture.
The system separates presentation, business logic, data management, and AI processing
into independent components.

\subsection{Expected Outcomes}

The expected outcomes of the proposed solution are summarized below:

\begin{itemize}
    \item Improved learning efficiency through adaptive quizzes.
    \item Reduced workload for teachers via AI-assisted question generation.
    \item Better learner engagement through interactive and real-time features.
    \item A scalable and maintainable system architecture suitable for future extensions.
\end{itemize}

\section{Report Organization}

\begin{table}[H]
\centering
\caption{Structure of the report}
\label{tab:report_organization}
\begin{tabular}{|c|p{10cm}|}
\hline
\textbf{Chapter} & \textbf{Description} \\ \hline
Chapter 1 & Introduction and project overview \\ \hline
Chapter 2 & Requirement analysis and system modeling \\ \hline
Chapter 3 & System design and architecture \\ \hline
Chapter 4 & System implementation \\ \hline
Chapter 5 & Testing and evaluation \\ \hline
Chapter 6 & Conclusion and future work \\ \hline
\end{tabular}
\end{table}
