\chapter{Introduction}

\section{Problem Statement}

\subsection{Background and Motivation}

\begin{figure}[H]
    \centering
    \includegraphics[width=\textwidth]{images/existing.png}
    \caption{Overview of limitations in existing quiz platforms}
    \label{fig:problem_context}
\end{figure}

With the rapid growth of online education, quiz-based assessment systems have become an essential
tool for evaluating learner performance.
Platforms such as Quizizz, Google Forms, and similar systems are widely adopted due to their ease
of use and accessibility.
However, these systems primarily focus on static assessment rather than supporting continuous and
personalized learning.

\subsection{Identified Problems}

Through analysis of existing solutions, several key limitations have been identified:

\begin{itemize}
    \item \textbf{Lack of adaptive difficulty:}
    Most platforms provide a fixed set of questions, ignoring individual learner ability and progress.
    
    \item \textbf{Limited learning feedback:}
    Feedback is often restricted to numerical scores, without explaining learning gaps or misconceptions.
    
    \item \textbf{Manual content creation overhead:}
    Teachers must invest significant time in creating and maintaining large question banks.
\end{itemize}

These problems reduce learning effectiveness and limit the long-term educational value of quiz systems.

\section{Objectives and Scope}

This section defines what the QuizSpark project aims to achieve and clearly delimits the boundaries of the system. First, the core pedagogical and technical objectives are summarized, focusing on adaptive assessment, AI-assisted content creation, enhanced feedback, and interactive learning. Then, the project scope specifies which functionalities and implementation concerns are included in this work and which related topics are explicitly excluded or reserved for future development.

\subsection{Project Objectives}

\begin{table}[H]
\centering
\begin{tabular}{|p{4cm}|p{9cm}|}
\hline
\textbf{Objective} & \textbf{Description} \\ \hline
Adaptive assessment & Adjust quiz difficulty based on learner performance \\ \hline
AI-assisted content creation & Automatically generate quiz questions from topics or documents \\ \hline
Enhanced feedback & Provide detailed result analysis and learning insights \\ \hline
Interactive learning & Support real-time quiz sessions and leaderboards \\ \hline
\end{tabular}
\caption{Objectives of the QuizSpark project}
\label{tab:intro_objectives}
\end{table}

\subsection{Project Scope}

The scope of the project is defined as follows:

\textbf{In Scope:}
\begin{itemize}
        \item Design and implementation of a web-based quiz system that supports core assessment workflows for students and teachers.
        \item AI-based question generation from teacher-provided topics or uploaded documents.
        \item Adaptive learning mechanisms that use each student's historical quiz results to automatically tailor question difficulty and content in subsequent quizzes.
        \item Dashboards and reports for performance analytics at both individual and class levels.
\end{itemize}

\textbf{Out of Scope:}
\begin{itemize}
        \item Advanced AI model research, such as training new large language models.
        \item Large-scale cloud cost optimization and infrastructure tuning for millions of users.
        \item Commercial deployment activities, including billing integration, marketing, and enterprise-level support. These are acknowledged as potential future work but remain outside the scope of this academic project.
\end{itemize}

\section{Software Development Model}

\begin{figure}[H]
    \centering
    \includegraphics[width=\textwidth]{images/waterfall.png}
    \caption{The Waterfall Model}
    \label{fig:intro_architecture}
\end{figure}

The QuizSpark project follows the \textbf{Waterfall development model} (Figure~\ref{fig:intro_architecture}).
System requirements are defined at the beginning of the project and the
development process is conducted through sequential phases, including
requirements analysis, system design, implementation, testing, and documentation.

Each phase is completed and validated before moving to the next one,
ensuring clear deliverables, stable outputs, and consistency throughout
the project lifecycle.

\paragraph{Expected Outcomes}

The application of the Waterfall model is expected to deliver:

\begin{itemize}
    \item A clearly structured and well-documented development process.
    \item Consistency between requirements, design, and implementation.
    \item A stable system that meets the defined project requirements.
\end{itemize}


\section{Report Organization}

This report is organized into six chapters, each focusing on a specific stage of the QuizSpark project lifecycle.

\paragraph{Chapter 1: Introduction and project overview}
Presents the problem context, project motivation, objectives, scope, and the selected Waterfall development model, giving readers an overall picture of why QuizSpark is needed and what the project aims to deliver.

\paragraph{Chapter 2: Requirements analysis}
Analyzes stakeholders and existing quiz platforms, identifies functional and non-functional requirements, and derives the key features that QuizSpark must support to address current limitations.

\paragraph{Chapter 3: Technologies used}
Describes the main technologies and tools adopted for the system, including the React-based frontend stack, the Spring Boot backend, database solutions, AI integration libraries, and supporting development tools.

\paragraph{Chapter 4: System design}
Explains the overall architecture of QuizSpark, such as the client–server structure, MVC-based backend organization, logical layers, and detailed component interactions between modules and services.

\paragraph{Chapter 5: Implementation, testing, and deployment}
Details how the designed architecture is implemented in practice on both frontend and backend, illustrates key modules in operation, and describes the functional testing strategy as well as the deployment process using containerization.

\paragraph{Chapter 6: Conclusion and future work}
Summarizes the main results achieved by the project, evaluates how well the objectives have been met, and outlines potential directions for improving and extending QuizSpark in future research and development.
